\documentclass[a4paper,12pt,twoside,openany]{memoir}
\setheadfoot{28pt}{1cm}

%%% DIVERSE PAKKER %%%
\usepackage{float}

% Listings til kode

\usepackage{listings}
\lstloadlanguages{[Sharp]C}
\renewcommand\lstlistingname{Code example}

\lstset{ %
language=[Sharp]C,              % the language of the code
%basicstyle=\footnotesize,      % the size of the fonts that are used for the code
numbers=left,                   % where to put the line-numbers
%numberstyle=\footnotesize,     % the size of the fonts that are used for the line-numbers
stepnumber=1,                   % the step between two line-numbers. If it's 1, each line 
                                % will be numbered
numbersep=5pt,                  % how far the line-numbers are from the code
%backgroundcolor=\color{white}, % choose the background color. You must add \usepackage{color}
showspaces=false,               % show spaces adding particular underscores
showstringspaces=false,         % underline spaces within strings
showtabs=false,                 % show tabs within strings adding particular underscores
frame=single,                   % adds a frame around the code
tabsize=2,                      % sets default tabsize to 2 spaces
captionpos=t,                   % sets the caption-position to top
breaklines=true,                % sets automatic line breaking
breakatwhitespace=false,         % sets if automatic breaks should only happen at whitespace
%title=\lstname,                % show the filename of files included with \lstinputlisting;
                                % also try caption instead of title
%escapeinside={\%*}{*)},        % if you want to add a comment within your code
%morekeywords={*,...}           % if you want to add more keywords to the set
}

% Dansk orddeling
\usepackage[english]{babel}

%Floats og [H]
\usepackage{float}

% Nummering af eksempler og sætninger
\usepackage{amsthm}
\theoremstyle{definition}
\newtheorem{example}{Example}
\newtheorem{theorem}{Theorem}

% Giver mulighed for at bruge æ, ø og å i .tex-filer
\usepackage[utf8x]{inputenc}

% Hjælper med orddeling ved æ, ø og å. Sætter fontene til at
% være ps-fonte, i stedet for bmp
\usepackage[T1]{fontenc}

% Pakke, der gør det muligt at anvende grafikfiler, samt andvende flere captions på samme figure
\usepackage{graphicx}
\usepackage{caption}
\usepackage{subcaption}

% Pakker, der kan udelades, hvis man ikke gør megen brug af matematik
\usepackage{amsmath}
\usepackage{amssymb}

% Pakke, der kan indsætte nødvendige mellemrum efter
% makro-anvendelser.
\usepackage{xspace}

% Retter forskellige bugs i LaTeX-kernen
\usepackage{fixltx2e}

% Indsæt rettelser og lignende med \fixme{...} med "final"
% i stedet for "draft" udløses en error for hver fixme,
% når der compiles.
\usepackage[footnote,draft,english,silent,nomargin]{fixme}

% Gør det muligt at definere farver.
% Se en.wikibooks.org/wiki/LaTeX/Colors for mere info.
\usepackage{color}
\usepackage[usenames,dvipsnames]{xcolor}

%Hyperlinks i indholdsfortegnelse
\usepackage{hyperref}

%Itemization i tables og andet leg
\usepackage{tabularx}
\usepackage{booktabs} % http://ctan.org/pkg/booktabs
\newcommand{\tabitem}{~~\llap{\textbullet}~~}

\hypersetup{
	bookmarks=true,  % Vis 'bookmark'-ramme.
	colorlinks=true, % True = links har farver, False = links har rammer
	citecolor=black,  % Linkfarve for referencer.
	linkcolor=black,  % Linkfarve i indholdsfortegnelse
	urlcolor=black,   % Linkfarve for eksterne URL'er.
}

%%% LITTERATURLISTE %%%
\usepackage[square,numbers]{natbib}
\usepackage{url}

%%% MARGINER %%%
\setlrmarginsandblock{3.5cm}{2.5cm}{*}	% \setlrmarginsandblock{Indbinding}{Kant}{Ratio}
\setulmarginsandblock{2.5cm}{3.0cm}{*}	% \setulmarginsandblock{Top}{Bund}{Ratio}
\checkandfixthelayout 			% Laver forskellige beregninger og sætter de almindelige længder op til brug ikke memoir pakker


%%% OTHER STUFF %%%
\newcommand{\pdf}{PDF}

\newcommand{\Z}{\ensuremath{\mathbb{Z}}\xspace}

% Kommando, der sikrer ensartede referencer til figurer

\newcommand{\figref}[1]{Figure \ref{#1}}

\newcommand{\mgd}[2]{\ensuremath{\{ #1 \mid #2 \}}}

%\newtheorem{saetning}{S�tning}
\newtheorem{definition}{Definition}




\usepackage{cleveref}
\crefname{listing}{code example}{code examples}
\usepackage{todonotes}

% Valg a kapiteludseende
\definecolor{numbercolor}{gray}{0.7}			% Definerer en farve til brug til kapiteludseende
\newif\ifchapternonum

\makechapterstyle{jenor}{									% Definerer kapiteludseende -->
  \renewcommand\printchaptername{}
  \renewcommand\printchapternum{}
  \renewcommand\printchapternonum{\chapternonumtrue}
  \renewcommand\chaptitlefont{\fontfamily{pbk}\fontseries{db}\fontshape{n}\fontsize{25}{35}\selectfont\raggedleft}
  \renewcommand\chapnumfont{\fontfamily{pbk}\fontseries{m}\fontshape{n}\fontsize{1in}{0in}\selectfont\color{numbercolor}}
  \renewcommand\printchaptertitle[1]{%
    \noindent
    \ifchapternonum
    \begin{tabularx}{\textwidth}{X}
    {\let\\\newline\chaptitlefont ##1\par} 
    \end{tabularx}
    \par\vskip-2.5mm\hrule
    \else
    \begin{tabularx}{\textwidth}{Xl}
    {\parbox[b]{\linewidth}{\chaptitlefont ##1}} & \raisebox{-15pt}{\chapnumfont \thechapter}
    \end{tabularx}
    \par\vskip2mm\hrule
    \fi
  }
}																						% <--

%\chapterstyle{jenor}
\usepackage{calc,color}
\newif\ifNoChapNumber
\newcommand\Vlines{%
\def\VL{\rule[-2cm]{1pt}{5cm}\hspace{1mm}\relax}
\VL\VL\VL\VL\VL\VL\VL}
\makeatletter
\setlength\midchapskip{0pt}
\makechapterstyle{VZ43}{
\renewcommand\chapternamenum{}
\renewcommand\printchaptername{}
\renewcommand\printchapternum{}
\renewcommand\chapnumfont{\Huge\bfseries\centering}
\renewcommand\chaptitlefont{\Huge\bfseries\raggedright}
\renewcommand\printchaptertitle[1]{%
\Vlines\hspace*{-2em}%
\begin{tabular}{@{}p{1cm} p{\textwidth-3cm}}%
\ifNoChapNumber\relax\else%
\colorbox{black}{\color{white}%
\makebox[.8cm]{\chapnumfont\strut \thechapter}}
\fi
& \chaptitlefont ##1
\end{tabular}
\NoChapNumberfalse
}
\renewcommand\printchapternonum{\NoChapNumbertrue}
}
\makeatother
\chapterstyle{VZ43}



\setsecnumdepth{subsection}		 					% Dybden af nummerede overkrifter (part/chapter/section/subsection)
\maxsecnumdepth{subsection}							% Ændring af dokumentklassens grænse for nummereringsdybde
\settocdepth{subsection} 							% Dybden af indholdsfortegnelsen


%%% FORSIDE %%%
%\title{Dette er rapportens titel}
%\author{Forfatter 1 \\ Forfatter 2}
%\date{November 2011}

% Sørger for LaTeX ikke strækker teksten ved at forhindre
% der bliver tilføjet linieskift, hvis siden ikke er
% fyldt helt ud.
\raggedbottom 


