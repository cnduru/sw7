\documentclass[a4paper,12pt,twoside,openany]{memoir}
\setheadfoot{28pt}{1cm}

%%% DIVERSE PAKKER %%%
\usepackage{float}

% Listings til kode

\usepackage{listings}
\lstloadlanguages{[Sharp]C}
\renewcommand\lstlistingname{Code example}

\lstset{ %
language=[Sharp]C,              % the language of the code
%basicstyle=\footnotesize,      % the size of the fonts that are used for the code
numbers=left,                   % where to put the line-numbers
%numberstyle=\footnotesize,     % the size of the fonts that are used for the line-numbers
stepnumber=1,                   % the step between two line-numbers. If it's 1, each line 
                                % will be numbered
numbersep=5pt,                  % how far the line-numbers are from the code
%backgroundcolor=\color{white}, % choose the background color. You must add \usepackage{color}
showspaces=false,               % show spaces adding particular underscores
showstringspaces=false,         % underline spaces within strings
showtabs=false,                 % show tabs within strings adding particular underscores
frame=single,                   % adds a frame around the code
tabsize=2,                      % sets default tabsize to 2 spaces
captionpos=t,                   % sets the caption-position to top
breaklines=true,                % sets automatic line breaking
breakatwhitespace=false,         % sets if automatic breaks should only happen at whitespace
%title=\lstname,                % show the filename of files included with \lstinputlisting;
                                % also try caption instead of title
%escapeinside={\%*}{*)},        % if you want to add a comment within your code
%morekeywords={*,...}           % if you want to add more keywords to the set
}

% Dansk orddeling
\usepackage[english]{babel}

%Floats og [H]
\usepackage{float}

% Nummering af eksempler og sætninger
\usepackage{amsthm}
\theoremstyle{definition}
\newtheorem{example}{Example}
\newtheorem{theorem}{Theorem}

% Giver mulighed for at bruge æ, ø og å i .tex-filer
\usepackage[utf8x]{inputenc}

% Hjælper med orddeling ved æ, ø og å. Sætter fontene til at
% være ps-fonte, i stedet for bmp
\usepackage[T1]{fontenc}

% Pakke, der gør det muligt at anvende grafikfiler, samt andvende flere captions på samme figure
\usepackage{graphicx}
\usepackage{caption}
\usepackage{subcaption}

% Pakker, der kan udelades, hvis man ikke gør megen brug af matematik
\usepackage{amsmath}
\usepackage{amssymb}

% Pakke, der kan indsætte nødvendige mellemrum efter
% makro-anvendelser.
\usepackage{xspace}

% Retter forskellige bugs i LaTeX-kernen
\usepackage{fixltx2e}

% Indsæt rettelser og lignende med \fixme{...} med "final"
% i stedet for "draft" udløses en error for hver fixme,
% når der compiles.
\usepackage[footnote,final,english,silent,nomargin]{fixme}

% Gør det muligt at definere farver.
% Se en.wikibooks.org/wiki/LaTeX/Colors for mere info.
\usepackage{color}
\usepackage[usenames,dvipsnames]{xcolor}

%Hyperlinks i indholdsfortegnelse
\usepackage{hyperref}

%Itemization i tables og andet leg
\usepackage{tabularx}
\usepackage{booktabs} % http://ctan.org/pkg/booktabs
\newcommand{\tabitem}{~~\llap{\textbullet}~~}

\hypersetup{
	bookmarks=true,  % Vis 'bookmark'-ramme.
	colorlinks=true, % True = links har farver, False = links har rammer
	citecolor=black,  % Linkfarve for referencer.
	linkcolor=black,  % Linkfarve i indholdsfortegnelse
	urlcolor=black,   % Linkfarve for eksterne URL'er.
}

%%% LITTERATURLISTE %%%
\usepackage[square,numbers]{natbib}
\usepackage{url}

%%% MARGINER %%%
\setlrmarginsandblock{3.5cm}{2.5cm}{*}	% \setlrmarginsandblock{Indbinding}{Kant}{Ratio}
\setulmarginsandblock{2.5cm}{3.0cm}{*}	% \setulmarginsandblock{Top}{Bund}{Ratio}
\checkandfixthelayout 			% Laver forskellige beregninger og sætter de almindelige længder op til brug ikke memoir pakker


%%% OTHER STUFF %%%
\newcommand{\pdf}{PDF}

\newcommand{\Z}{\ensuremath{\mathbb{Z}}\xspace}

% Kommando, der sikrer ensartede referencer til figurer

\newcommand{\figref}[1]{Figure \ref{#1}}

\newcommand{\mgd}[2]{\ensuremath{\{ #1 \mid #2 \}}}

%\newtheorem{saetning}{S�tning}
\newtheorem{definition}{Definition}




\usepackage{cleveref}
\crefname{listing}{code example}{code examples}
\usepackage{todonotes}

% Valg a kapiteludseende
\definecolor{numbercolor}{gray}{0.7}			% Definerer en farve til brug til kapiteludseende
\newif\ifchapternonum

\makechapterstyle{jenor}{									% Definerer kapiteludseende -->
  \renewcommand\printchaptername{}
  \renewcommand\printchapternum{}
  \renewcommand\printchapternonum{\chapternonumtrue}
  \renewcommand\chaptitlefont{\fontfamily{pbk}\fontseries{db}\fontshape{n}\fontsize{25}{35}\selectfont\raggedleft}
  \renewcommand\chapnumfont{\fontfamily{pbk}\fontseries{m}\fontshape{n}\fontsize{1in}{0in}\selectfont\color{numbercolor}}
  \renewcommand\printchaptertitle[1]{%
    \noindent
    \ifchapternonum
    \begin{tabularx}{\textwidth}{X}
    {\let\\\newline\chaptitlefont ##1\par} 
    \end{tabularx}
    \par\vskip-2.5mm\hrule
    \else
    \begin{tabularx}{\textwidth}{Xl}
    {\parbox[b]{\linewidth}{\chaptitlefont ##1}} & \raisebox{-15pt}{\chapnumfont \thechapter}
    \end{tabularx}
    \par\vskip2mm\hrule
    \fi
  }
}																						% <--

%\chapterstyle{jenor}
\usepackage{calc,color}
\newif\ifNoChapNumber
\newcommand\Vlines{%
\def\VL{\rule[-2cm]{1pt}{5cm}\hspace{1mm}\relax}
\VL\VL\VL\VL\VL\VL\VL}
\makeatletter
\setlength\midchapskip{0pt}
\makechapterstyle{VZ43}{
\renewcommand\chapternamenum{}
\renewcommand\printchaptername{}
\renewcommand\printchapternum{}
\renewcommand\chapnumfont{\Huge\bfseries\centering}
\renewcommand\chaptitlefont{\Huge\bfseries\raggedright}
\renewcommand\printchaptertitle[1]{%
\Vlines\hspace*{-2em}%
\begin{tabular}{@{}p{1cm} p{\textwidth-3cm}}%
\ifNoChapNumber\relax\else%
\colorbox{black}{\color{white}%
\makebox[.8cm]{\chapnumfont\strut \thechapter}}
\fi
& \chaptitlefont ##1
\end{tabular}
\NoChapNumberfalse
}
\renewcommand\printchapternonum{\NoChapNumbertrue}
}
\makeatother
\chapterstyle{VZ43}



\setsecnumdepth{subsection}		 					% Dybden af nummerede overkrifter (part/chapter/section/subsection)
\maxsecnumdepth{subsection}							% Ændring af dokumentklassens grænse for nummereringsdybde
\settocdepth{section} 							% Dybden af indholdsfortegnelsen


%%% FORSIDE %%%
%\title{Dette er rapportens titel}
%\author{Forfatter 1 \\ Forfatter 2}
%\date{November 2011}

% Sørger for LaTeX ikke strækker teksten ved at forhindre
% der bliver tilføjet linieskift, hvis siden ikke er
% fyldt helt ud.
\raggedbottom 



\begin{document}

%\include{formalia/forside}
%\cleardoublepage

%\frontmatter
%%\Blankpage
\phantomsection
\thispagestyle{empty}
% Titlepage [START]
    \sectionmark{Titlepage}

    \begin{tabular}{r}
        \parbox{\textwidth}{\raisebox{-15mm}{\includegraphics[height=3cm]{billeder/aauLogoEnStudent.png}} %
         \hfill \parbox{4.9cm}{ %
            \begin{tabular}{l} 
                {\textsf{\small{\textbf{Department of Computer Science}}}}\\
                {\textsf{\small{\textbf{Software 7th semester}}}}\\
                {\textsf{\small{Address: Selma Lagerlöfs Vej 300}}} \\
                {\textsf{\small{\hspace{14 mm} 9220 Aalborg Øst }}} \\
                {\textsf{\small{Phone no.: 99 40 99 40}}} \\
                {\textsf{\small{Fax no.: 99 40 97 98}}} \\
                {\textsf{\small{Homepage: \url{http://www.cs.aau.dk}}}}
            \end{tabular}}}
    \end{tabular}
    
    \begin{tabular}{cc}
	
        \parbox[3cm]{7cm}{ %
	\vspace{7mm}
            \begin{description}
                \item {\textbf{Project title}:} \\
                    ...
                    \hspace{4cm}
                \item {\textbf{Subject}:} \\
                    ...
            \end{description}
	\vspace{-4mm}
            \parbox{8cm}{ %
                \begin{description}
                    \item {\textbf{Project periode}:} \\
                        Autumn 2014
                    \hspace{4cm}
                    \item {\textbf{Group name}:} \\
                        sw702e14
                    \hspace{4cm}
                    \item {\textbf{Supervisor}:} \\
                        Ivan Aaen
                    \item {\textbf{Group members}:}\\%\newcommand{\sh}{18pt}\\%
                    Casper Holst Laustsen\\[0.20cm]
                    Christoffer Nduru\\[0.20cm]
                    Dan Skøtt Petersen\\[0.20cm]
                    Johan Leth Gregersen\\[0.20cm]
                    Kristian Mikkel Thomsen\\[0.20cm]
                    Morten Møller Jakobsen
                \end{description}
            }
	    \vspace{-4mm}
            \begin{description}
                \item {\textbf{Copies}:} ...
                \item {\textbf{Pages}:} \pageref{LastPage}
                \item {\textbf{Appendices}: ...} 
                \item {\textbf{Finished}: ...} 
            \end{description}
            \vfill 
        } &
        \parbox{7cm}{ %
            %\vspace{.15cm} %
            \hfill %
            \begin{tabular}{l}%
                {\textbf{Abstract}:}\bigskip \\%
                \fbox{ %
                    \parbox{6.2cm}{\bigskip %
                    {\vfill{\small %
                     %Purpose: To give readers a quick identification of the basic content of the thesis. It should “stand on its own” and be a self-contained document. There should be no need to look elsewhere in the thesis for an understanding of what is said in the abstract.
%1: Objectives and scope
%2: A description of the methods used
%3: A summary of the results
%4: A statement of the main conclusions

In this project it is attempted to develop a general server framework that developers can utilize when creating multi-player games. The goal is to be flexible enough that the framework can be used to support a wide array of games, but still provide required functionality to save developers the time of developing game specific frameworks themselves. To help demonstrate the developed solution, an Android application is developed simultaneously, utilizing the server functionality. At the end of the project, both the framework and application are functional, but not complete. The server framework has been tested in stability and load, and was found to be reliable. The developed solution has however ended up being less separated than desired, supporting the Android application a little too specific.
                        \bigskip}}%
                    }}%
            \end{tabular}%
        }
    \end{tabular}

    \noindent{\footnotesize\emph{The material in this report is freely and publicly available, publication with source reference is only allowed with authors' permission.}}
% Titlepage [END]
%\Blankpage

%\cleardoublepage


% De vigtigste forskelle mellem \include og \input ligger i at
% \include kun må findes EFTER preamblen og at 
% inkluderede filer kan fravælges ved generering af output
% ved brug sammen med \includeonly.

% Se arbejdsblad-skabelon.tex for et eksempel herpå.

%\thispagestyle{empty}
\section*{Preface}
This report was written by six 7th semester software engineering students from Aalborg University.\\

\noindent The report documents the development of a framework for meant for developing location-based multiplayer games.

\noindent The source code for this project including raw test data is accessible through the following link:
\begin{itemize}
\item \url{http...}\fxfatal{Preface: Link til source code.}
\end{itemize}

\vspace{.2cm}
\noindent We would like to thank our supervisor Ivan Aaen for providing guidance and feedback.

\begin{table}[H]
	\centering
	\vspace{2cm}
		\begin{tabular}{c c c}
			\underline{\phantom{JAERJAERJAERJAERGO}} & \phantom{cookies} & \underline{\phantom{JAERJAERJAERJAERGO}} \\
			Casper Holst Laustsen & \phantom{cookies} & Christoffer Ndũrũ\\[1.5cm]
		    \underline{\phantom{JAERJAERJAERJAERGO}} & \phantom{cookies} & \underline{\phantom{JAERJAERJAERJAERGO}} \\
			Dan Skøtt Petersen & \phantom{cookies} & Johan Leth Gregersen\\[1.5cm]
			\underline{\phantom{JAERJAERJAERJAERGO}} & \phantom{cookies} & \underline{\phantom{JAERJAERJAERJAERGO}} \\ 
			Kristian Mikkel Thomsen & \phantom{cookies} & Morten Møller Jakobsen\\[.5cm]				
		\end{tabular}
\end{table}


% For at sikre sideskift efter forord. Denne kommando bør kun bruges i
% absolutte undtagelsestilfælde.

%\newpage
\mainmatter
\tableofcontents*
\newpage

%Main parts of report
\listoffixmes
%INTRODUCTION
\chapter{Introduction}
\label{chap:intro}
%http://thesistips.wordpress.com/2012/03/25/how-to-write-your-introduction-abstract-and-summary/

% It introduces the problem and motivation for the study.

% - Tell the reader what the topic of the report is.
% - Explain why this topic is important or relevant.

% It provides a brief summary of previous engineering and/or scientific work on the topic.
% - Here you present an overview what is known about the problem.  You would typically cite earlier studies conducted on the same topic and/or at this same site, and in doing so, you should reveal the yawning void in the knowledge that your brilliant research will fill.

% It outlines the purpose and specific objectives of the project.
Development of an online multiplayer game serving many users can be split into two parts: Creating the explicit game functionality, and creating underlying architecture to support it. When developing new games this is often mixed together as the underlying architecture is only used for the specific game. This gives the developer the opportunity to write an extremely game specific architecture. In some cases, the developer may want to create other similar games with the same underlying structure. However, as the architecture and the game itself is intertwined, it is hard to reuse the same structure. The developer would have to first search through the previous model and extract the desired features, wasting some time on this. Furthermore, the functionality might not be an exact match to what the current needs, requiring additional time spent to adjust the code.

The focus of this project is to create a flexible and reliable architecture for a server that supports client-based multiplayer games. The goal is to allow easy development of multiplayer games for smartphones, and to make it possible for developers to focus on the game development without having to worry about managing the underlying architecture. 

The framework will focus on creating a platform for location-based games. It has to be scalable, meaning the server can easily handle both multiple games and multiple devices (players) at the same time. This means it is a requirement that multiple clients should be able to connect to the server and perform different actions at the same time. There is a focus on creating a thread based server framework where these threads handles actions individually. The framework should have the functionality of handling GPS-coordinates, using it as the base of some sort of action, depending on the game implementation. 

Creating new games from scratch requires an investment in both time and money. And while some firms might have these remedies available, it could increase the potential profit if this time could be saved. The extra time either means a bigger surplus for the developer or more time to work on the game features. We argue that creating a precise but flexible framework can help improve this situation.

A framework is the basic set of functionality and structural guidelines for the game to be built upon. Overall game development rarely fits typical software life cycle methods such as the waterfall method as requirements can change quickly. Fairly, an agile development process is often used\cite{Gamedevelopment}. The result is that functionality might have to change on a frequent basis. Whether this means new functionality or rewriting already existing functions - a game with an intertwined framework and game will take more time to rewrite, as components often have or provide dependencies for other parts. Consider a third person shooter game where functionality revolves around specific gun types and their functionalities. Far into the development, another gun type might be a new requirement. The developer would, depending on the existing framework have to rewrite much of the code regarding guns in general to suddenly include new functionality. This requires a lot of extra time compared to having a framework that allows fluent adding and removal of gun functionality. Worse is examples where you have to abandon an idea and create something entirely different. Having to rewrite all the code would be tedious, especially when it involves the framework. Having an independent framework could help alleviate this issue.

The size and the amount of divisions of a developing firm also matters. The more different people working on it, the more precision is needed both in the documentation and the framework. Having multiple people writing code on the same game requires the underlying architecture to be as steady and consistent as possible. Changing something in the framework or the underlying architecture requires a notification to all different divisions and people working on the game. It might result in some code refactoring \cite{Gameprod}. These firms often create their own framework and document it through the game design document \cite{Gamedesign} in order for it to fit exactly their needs. This causes them to have a consistent and steady framework to work upon, and work individually without consulting different divisions. Smaller developers might not have the luxury of creating a completely game specific framework for each of their projects. This results in smaller developers often having to spend more work hours adjusting their code to what their co-developers have coded.

Another problem with developing different games from scratch as opposed to having a standardized framework is the compatibility. Having different types of devices with different versions of various operating systems means that some devices might not be compatible with the desired program. What this means is that programs act differently on different devices. With an existing framework this only have to be implemented once. Having a framework that is already compatible with the different devices allows the coder to focus solely on the game and therefore limiting the amount of time needed for creating versatile code.

Constructing a framework that could be used by different developers could save the developer a lot of time and money, given the developer uses the framework as intended. It could and should be easier to use for the developer - giving him less work hours needed to create the same content he otherwise would. The struggle with creating a framework like this is that it has to be flexible enough for game developers to create diverse games, but still be precise enough to not having to create an immense amount of code in order to fit it to the game.

The subject is widely pursued in the form of game engines and general APIs. A large amount of games being developed are developed on an already existing game engines. An example of a game engine is the `Adventure Game Studio', providing a sizable API for creating games in third person perspective along with support for pre-rendering. The developer, Chris Jones, have created different functionality he deemed important for creating these games and published it in the form of a simple IDE for game developers \cite{adv-game}. It is specific in the way its focused on third person adventure games, making specific functionality.
Another example of game engines is full blown engines with multiple purposes for a specific genre. One of the most used engines for first person shooters is the "Unreal Engine" which is used for many major shooter or fighter games. It is used both as visual aid, providing 3D rendering, but also provides pre-rendering functionality.

% It provides a ‘road map’ for the rest of the report.
% - This is so that the reader knows what’s coming and sees the logic of your organization.
% - Describe (in approximately one sentence each) the contents of each of the report/thesis chapters.

\section{Problem Statement}
\label{sec:probstate}
When creating games the amount of time and funds spent are valuable resource. To ensure a high amount of quality code, the developer has to spent a lot of time writing trivial code and it will thrive the cost up. This naturally raises the question:

\mbox{\parbox[b][2.5cm][c]{0.95\textwidth}{\textit{How can we develop a framework for creating multiplayer games - increasing the experience for the developers}}}


\fxfatal{Kan det skærpes og generaliseres? - Ivan Aaen 2014 - 9/12}
The idea is simple, we want to creating framework for creating simple coordinate-based multiplayer shooter games. In order for the problem statement to be fulfilled, we raised some additional goals to the question asked above:

\begin{itemize}
\item The framework has to be as flexible as possible.
\item The framework has to account for multiple instances of games with multiple users.
\item The framework has to be easy to use. 
\end{itemize}


%Scenarios
%\section{Scenario 1: Maps \& Navigation - City areas}
In a large city, e.g. one of the 5 largest cities in Denmark, data connectivity can be obscured by many big buildings or concrete parking garage. This type of data connection loss can be unpredictable and happen in an instant. A scenario can be a travelling salesman arriving at an unknown large city trying to find a customer on a navigation system. The salesman is limited on time and relying completely on the navigation system to find him the customer. The salesman will rely on a system that can detect which buildings will cause connection loss and hereby download maps and navigation in time.
%\section{Scenario 2: Maps \& Navigation - Countryside Areas}
When travelling out of cities, data connectivity can be unavailable for a longer period of time. This type of data connection loss is slowly fading away as you move further away from the connection source.

A scenario can be a couple going for a walk in the woods on the countryside. They are visiting an unknown area and walks into the woods. They happen to get lost and wants to find their way back to the car, but in the woods no data connection is available and they are lost.

A modification of this scenario could be a couple that wants to get additional knowledge of the area they are walking around in, e.g. visiting Rold skov in Denmark, they might want to read the stories about the robbers from Rold. It is also possible that they want to know about the attractions in the area, and directions to them.

The challenges are predicting that the data connection will be lost in the near future, and what services should be attempted to download to a mobile device. 

%\section{Scenario 3 Bus and Train}
In this scenario the problem be solved is to minimize Internet dropouts when traveling on frequently used bus or train lines. The idea is to collect data on a user basic and use that data to identify areas where the user might lose connection.

The solution should be able to detect traveling by bus or train and predict the route and destination. It should reduce the user experience of no Internet doing the travel and make relevant information available for the destination should it be in a poorly connected area.



%In order to do so several approaches are available.

%A simple approach is to measure the signal strength of the device and use that as the base for predicting when connection will be unavailable. Using this method a stable but slowly dropping signal strength might indicate a future with no connection available, in this event the required web resources should be cached so they are available doing an offline period. Another scenario could when the signal is unstable but the average remains high, this could indicate interference in the area but no risk of a significant offline period, in this case caching is not necessary.
%
%Another approach is to base the prediction on historical data of the near area. This can be done by gathering data of where Internet connections usually drops and build a map for the available providers. Then the caching will be done when a device is predicted to move into such an area.

%Problem statement
% How can a user be guaranteed to always have relevant information available, regardless of network connectivity?
\section{Chapter Descriptions}
The following is a brief description of each chapter in the report

% add more here as report progresses %
\paragraph{From Idea to Framework}
This chapter will give a general idea behind the choices made - why we chose to make a game in addition to the framework and what outcome we expect to get from it. In addition there will be a short description of the underlying architecture and goal of the framework. 
\paragraph{The Game}
This chapter will describe the game we used as foundation for creating the framework. It will describe the client, how it was designed, and lastly how it was implemented.
\paragraph{The Framework}
This chapter describes all of the choices considered for the framework. It will describe how the game influenced the server and framework, and advantages and disadvantages of this. Furthermore it will describe design and implementation of the server and database. Lastly, a section will describe tests and the result of it.
\paragraph{Conclusion}
A brief summary of the entirety of the project and an evaluation of to which degree the problem statement was fulfilled.



%PROBLEM ANALYSIS
\chapter{Problem Analysis}

%% Problem
%  Define the problem in terms of:

%  - Problem area / General problem
%  - People affected
%  - Problem results in...
%  - What is gained from solving it?
Development of an online multiplayer game serving many users can roughly be split into two parts: Creating the game, and creating the underlying architecture. The focus of this project is on creating a flexible and reliable server architecture \fxfatal{client-server? smth else?} allowing easy development of multiplayer games for Android smartphones. The aim is to make it possible for developers to focus on the game development without having to worry about managing the underlying architecture. 

The main problem for new game developers is the issue of having to creating  good usable framework for their specific needs. At the start of a project, game developers set out to create a framework that can handle all of the functions that the game need, without really having specified precisely how those functions are implemented. This means that if functions change or something entirely new comes up, the framework have to be adjusted to the specific need. An example of this could be how the game handles several users, maybe the game have different needs depending on some variating variable. The amount of workload a developer has to put into creating the framework to match exactly his demand is greatly increased. The different requirements and often change in the framework often gives an everlastingness to the code.

It requires a lot more work and overview from the small game developers to keep the framework completely up to date. Bigger firms almost always require multiple sections to use the same framework - and therefor consistency is a must. More often than not these firms create their own frameworks to design them exactly the way they need it, but smaller firms might have limited funds. This means that smaller developers often either skip on creating features they normally would or create sloppy and inefficient code.

Creating a joint framework that could be used by different could save smaller and new game developers a lot of time, and create a standard that makes coding on top of it easy. \fxfatal{Dette er en start på problem analyse - der kan sagtens addes mere omkring potentielle problemer og non server related issues}




%% Potential problems
%  Server related:
%  - Network connection
%  - Efficient communication
%  - Many users
%  - GPS, latency (actions close to each other?)

%  Non server related:
%  - Attracting users
%  - Keeping users
%  - Earning money
%  - Similar applications to learn from

%% Constraints
%  Which contraints are there in regards to this project?


%DESIGN
%Chapter
\chapter{Design}
This chapter outlines the different design choices made for the project. In addition, possible alternatives are presented and discussed where applicable.
%Sections
\subsection{Blocking vs. non-blocking I/O}
For the client/server socket communication, a choice between blocking (synchronous) and non-blocking (asynchronous) I/O has to be taken.
% % blocking % %
Blocking I/O can have a better performance than non-blocking, but can cause problems when using a threaded architecture that spawns a new thread for each client.
% % non-blocking % %
Non-blocking I/O is chosen for this project. It scales well when there are many clients, and the system should scale well with an increasing number of clients. The ability to scale well does not come for free, however. Non-blocking IO is not always as fast as blocking IO and this can result in decreased performance. 
\section{Model View Controller}
\label{sec:mvc}
%We chose MVC - describe it

%What do we gain from MVC? What are the tradeoffs?

%Other options? Model-View-Presenter, Observer, Multitier architecture?

%Is it possible to change the architecture?

MVC is very widespread, and aside from often being misunderstood, there are also many subtle variations of it. To make sure we are on the same page, here is a quick outline of how we understand it:

\paragraph{Model} The model is a representation of “some information about the domain” \cite{fowler}. It can be implemented in different ways, but in essence it consists of data, business logic (e.g. validation or calculations), and some sort of interface to reach said data and logic from the other layers. The model in our case would be the entire server side, and a tiny bit on the client side. The server part of the model will be everything from database to sending and receiving information between server and client. On the client side the model part would be the specific functionality dedicated to setting up the connection.

\paragraph{View/Controller} \fxwarning{ref til dette i client design}This is the presentation layer. The view displays the content in the UI, but cannot/is not allowed to manipulate it. This job is done by the controller, which will take user inputs, manipulate the mode, and update the view accordingly. Because it is closely linked to the view, it is normal to have one controller per view. Given we implement a client side inform of a mobile device, the view should strictly be the GUI on the device. The controller should be implemented on a client side, adapting the view to whatever the server has sent back from the client request.\\

For each layer, there are a number of possible design patterns that can be applied. If you work with a framework these often already implement some sort of design pattern, but if you are working with plain MVC these patterns are important to structure your code if it has to live up to the quality requirements of a long term / large scale application.




\section{Design Patterns}
When considering how to design a software solution it can be a good idea to consider using some design patterns. Without those, code easily becomes messy later on in the project. Patterns can help by providing conventions and best practices to use as guidelines. With defined patterns it can be ensured that everyone working on the project has similar understandings of the code. It is worth to note that like everything, patterns can be misused; It is important to choose those that will fit the project, rather than try to force harmful patterns on to it. 

Below is a description of patterns found usable in this project. Patterns are split into two, one with patterns applicable in the client application, another with patterns applicable on the server program.

\subsection{Patterns in the application}
%\textbf{Singleton} was an obvious pattern to use in the application for certain elements. Firstly, the application including all activities are singleton. This is enforced by Android. There is no reason why the user should run multiple instances of the game. A certain class called "Client" is however also important to keep singleton. This class is responsible for opening a connection to the server and handle communication. It should not be allowed for a user to send multiple requests to the server at once.

\textbf{Lazy Initialization} pattern is designed to save/delay use of resources by postponing steps for as long as possible (Until right before it is needed).The Android application in this project does not perform many heavy operations, but it should always be remembered that the application is running on a phone, and that it has a battery. As such, most initializations of objects has been pushed back to the point where they are required along with operations on data.

\textbf{Adapter} pattern is a simple idea, but can be used in many contexts. It stems from the problem of wanting two incompatible interfaces to work together. The pattern solution is to use adapters to facilitate a common ground for the two interfaces. In Android, adapters are used to construct lists, grids and more. The adapters help link lists of items together with a visualization, using a given layout to inflate and a set of items. It will then provide a view of each item, effectively providing a user-friendly way of using/viewing the data. For this project adapters have been used to create and display clickable lists.

\textbf{Facade} pattern describes the idea of hiding messy procedures by using a facade. The result is that whoever wants to use the procedures have access to simple, easy to use facades which acts as access points to the procedural steps. In this project it has been used mostly for abstracting away the server communication. As this is done by creating XML strings with relevant data, this would be tedious to do in every activity that needed server communication. The communication itself was also abstracted. Instead, this functionality has been written in separate classes and is accessed through simple method calls. An example can be seen in \ref{fig:facadepattern}. This displays what happens when a user tries to login. The username and password is converted to XML and sent to the server, which in turn provide a responde. All the details of how this happens are however hidden.

\begin{figure}[H]
\centering
\includegraphics[width=\textwidth]{billeder/facadepattern.png}
\caption{Abstraction of server communication}
\label{fig:facadepattern}
\end{figure} 

\subsection{Patterns on server}
% If you are about to write about thread pool, we are not using a thread pool. We just make a thread for each game.

% for pattern in project:
	%Describe each pattern we use:
	% - What is its purpose?
	% - Pros/cons?
	% - Alternatives?
	% - Flexibility (refactoring, adding new functionality, changing the pattern...)
\subsection{Noget server.}
This section will contains possible improvements upon the current implementation. Some general improvements such as more are 

\subsubsection{Multi Threading}
In our implementation of the framework we have used threads and this has lead to a some unresolved issues. An example of this is in the dispatcher, see \Cref{sec:dispImplementation}, where threads gives problems when creating or closing games. This could be solved by restricting access to \texttt{GameThreadPool} which is currently shared amount all worker threads.

Another issue is that there is no set limit how many games can run at the same time. Eventually exceptions will be thrown when continuing to generate game threads.


\subsubsection{Blurriness Between Client and Sever}
In the implementation of the game not all method where implemented generically. An example is the callable method \texttt{Shoot} and the class weapon. Methods like \texttt{Shoot} should have been implemented with a \texttt{UseItem} call according to our design.

\subsubsection{Error Handling}
Error handling is sparse in some areas 

\subsubsection{Trusty Server}
everything the client sends is legal
\subsection{Interfaces}
\label{subsec:interfaces}
%INTRO:
%Server and client both consist of several small parts that all need to communicate.
%Examples: user->client, client->server, game->db...

We have decided that all information between the server and the client are shared in the form of XML. We have chosen to do so because of the diversity XML provides, and its interoperability with a wide variety of software. This fits well within the theme of creating a customizable framework for creating games. Furthermore XML is a widely used transport format for data, and thus many people know how to use it. 

The idea behind XML is to create tags for each different type of parameter. XML often fits well for sending information for games since it is flexible and extensible, this means there is no limitations to what the tags are and how many there are. Furthermore the tree structure of XML helps with encapsulations and scalability - as it is easy to emulate games in XML code. XML furthermore have schema support meaning it has the ability to specify the formats of a document, and for the receiver to check whether they are matching the format. \cite{xml}

An alternative to XML is JSON which is in many ways superior to XML as it is more compact and have better distinction between data types. However it lacks the schema and namespace support. XML is the dominant format for things like web services and file formats.

XML provides a convenient way to deserialize the input - making it an appropriate solution for small constant interactions like the ones we need. There are many other alternatives to XML, but we feel that it is a simple interaction without extraordinarily complicated needs - which makes the simplicity of XML a great trait for us.

When implementing a game with many different actions, it is needed to have different kinds of data to sent depending on which activity thats taken. The idea we have, is to have individual methods for each different scenario rather than creating a generic XML converter because we want to enforce the right data structures and types. 

The information is structured around what the server needs for authentication - so what the client sends is of course built around what the server requires. Since the frameworks functionality is a fixed size, this requires a fixed amount of methods on the client side.

For the same reasons mentioned above and for the sake of consistency the client expects XML on roughly the same form. This means the server side has to send information as well.

\section{Game Rules}
%Why did we make game rules?

%What are pros and cons of making them? (aka how do they help us, how do they limit us?)

%Write the game rules.

%Discuss how easily a different set of rules could be implemented
\section{Seperation of Framework and Game}
%It should be possbile to build a new game on the same framework.
The framework and the game is split into two completely seperate components. This enables creation of a new game on the same framework. 
%Why is the framework and game seperated?
The reason for this choice, is that the framework can be reused for different games. Because of this future developers who would like to make a game can reuse the framework and do not have to touch any framework code, only the game code itself. 
%Why is it a good idea, and what are the tradeoffs?
Developing the framework and game as two completely independent components has a number of good side effects. First of all, it makes it easier to avoid writing code which is intertwined (between the framework and the game). 
%What would happen if it is not split up?
If this was not avoided, it would be very hard to create an independent framework, since part og the game which was developed on top of the framework would be tied into the framework and vice versa. Therefore, a clear code-wise distinction between framework and server produces cleaner, more independent overall program structure. 
%Discuss the constraints of the framework (e.g. database schema, efficiency).
Because the framework and game are separated components, the framework has some limitations. The framework essentially enables development of many kinds of networked multiplayer games. Therefore, the database's tables for example have to be quite ``fixed'', since the developers of the games for the framework should not need to change the database and framework code. A database change will likely warrant some code in the framework to be rewritten. If the developers have to start rewriting code in the framework to develop their game, the point of the framework is lost. 

%\chapter{Prototypes}
\label{chap:proto}

\section{Handlers}
\label{sec:handlers}

\subsection{Packaging}
\label{sec:packingadv}

A solution to losing network services and therefore the possibility to download maps, is to detect that the device is losing network-connection in time. Hereby the device may start downloading maps over the nearby area and store for use once the network connection is lost. These maps should then be compartmentalized into groups with different prioritizations, based on a criteria of nearest-package-first. 

Downloading the by nearest-package-first requires an indexation of map-packages for a given location of a device. Since our service are to provide maps even when network connection is lost, we intend to use a map-service like Google Maps and not develop our own. Google Maps already divided a map into small packages, which are downloaded and displayed upon request depending on simple GPS-coordinates. New packages will be downloaded and displayed when the user moves the map on a mobile-device. The problem we deal with, are which packages to download without knowing more than GPS-coordinates. 

We put forward three solutions to this problem:
\begin{itemize}
\item Define a circular perimeter around the GPS-coordinates. Start downloading map-packages on the device GPS-coordinates, then prioritize after nearest-package-first until the perimeter is covered with downloaded packages.
\item Define a rectangular perimeter around the GPS-coordinates and proceed as above.
\item Process the GPS-coordinates over time, which will provide direction and velocity of the device. This should be used to download packages in the direction that the device is moving, and disregard the package behind the device. Considering the velocity, the width of required packages can be defined and as velocity increase packages further away can be downloaded. This solution does not follow nearest-package-first defined on first set of GPS-coordinates but rather try to dynamically follow nearest-package-first based on time and movement.
\end{itemize}




\begin{table} [h]
   \begin{center}
   \begin{minipage}{\textwidth}
      \centering
      \begin{tabularx} {\textwidth} { X | X  }
         \hline
		 & \\
         Advantages & Disadvantages \\
		& \\\hline
		& \\
         \tabitem adv & \tabitem disadv\\
         \tabitem adv & \tabitem disadv \\
		& \\\hline
      \end{tabularx}
      \caption{this is a dummy}
      \label{tab:packingadv}
   \end{minipage}
   \end{center}
\end{table}
\newpage

\section{Supplements}
\label{sec:supplements}


%\chapter{Architechture}
\label{chap:arch}


%THEORY
\chapter{Transmission Control Protocol}
The Transmission Control Protocol (TCP) is a part of the Internet protocol suite. Together with the Internet Protocol (IP), TCP is so widely used that the entire Internet protocol suite, containing many protocols, is often just called TCP/IP.

In the following, the parts of TCP that are particularly important for this project are described. This chapter is based on \cite{wiki-tcp}. \fxfatal{Find better source.}
%Author: Dan, Christoffer
\section{Data Transfer}
%Reliable transfer, error detection
Because we are working with transfer of data to a device with an uncertain connection, it is relevant to consider how TCP handles transferring data. An important aspect of this is to ensure reliable transmission, i.e. to make sure all the data is transferred correctly. To ensure this, each byte of data gets a sequence number, allowing the destination host to reconstruct the data in case of for example packet loss. Additionally, when a packet is received, the receiver sends back an acknowledgment, and if such an acknowledgment is not received the packet will be sent again. To ensure correctness of the packet content, each packet has a checksum included which the packet's content can be compared to upon arrival at its destination.\\

This possibility of error checking comes in handy in our project. When a map is transmitted to a client, it ensures that the map data is not corrupted. This is very desirable since corrupted map data could make the map unusable in the best case, and show wrong map data in the worst case.

\subsection{Flow Control}
It is possible for the server to sent more data than the receiver can process, and to accommodate this a flow control protocol is used. When data is sent, the receiver answers back with a \textit{window size}, telling the sender how much more data it is able to process. If the window size reaches 0, a persist timer will be set to account for the possibility that the updated window size was simply lost. When the timer runs out the server will send a small packet, probing the receiver for an updated window size.

%Should we mention anything about the header overhead?

% Kilde: http://en.wikipedia.org/wiki/Transmission_Control_Protocol#Data_transfer

\subsection{TCP/IP versus UDP}
We consider two aspects of TCP/IP versus UDP. The first aspect is which protocol is cheapest in space usage when transferring from one unit to another. The second aspect is which protocol offers the best suited services for our application.

These questions are related to the problem we try to solve. We want to try to download data on an endangered connection or download within a limited time-span before the device goes offline. Therefore we investigate TCP/IP versus UDP further.

When transferring data from one unit to another, the size of data to send varies depending on the program. However, the header has a minimum or fixed size it always uses, without considering the size of the data the user wants to send. In TCP/IP, the size of the header is 21 octets\cite{tcpdesc}. In UDP, the size of the header is 8 octets\cite{udpdesc}. In comparison, UDP uses 38\% of what TCP/IP uses for each exchange.

The services provided by TCP/IP are reliability through error checking and delivery validation whereas UDP emphasizes low-overhead operation and reduced latency\cite{wiki-tcp}. Both protocols provide valid services to be used when solving our problem. TCP/IP will ensure we have correct, consistent and working data. UDP could be used for transferring GPS-coordinates because they will be transferred extensively when the application is running. Losing an UDP transaction will not break the application. If a few data is lost or inconsistent, new data will be sent in the very near future.


%\section{Location Detection and Predicting Signal Loss}
To predict when a device might suffer a signal loss and become disconnected from the internet, two different scenarios must be considered. The scenarios are as follows\\

\begin{itemize}
\item The device is moving into the country side
\item The device is in the city\fxwarning{should this be removed? is the scenario not needed anymore?}
\end{itemize}

In order to predict when the signal may be at risk of being lost in the first scenario, it makes sense to look at measurements from the antenna of the device. A number of measurements, five for example, can then be stored on the device. These values should be error corrected (e.g. using standard deviation) to make sure sudden fluctuations will not trigger an alarm, telling the device that it has lost its signal. Since coverage is usually worse outside of big cities compared to in the cities, the signal will be lost gradually and not suddenly. Therefore it makes sense to measure a number of antenna measurements, and decide whether they are falling at a rate which would indicate heading towards the country side.\\

This functionality uses the moving average, which has the purpose of smoothing out temporary fluctuations in measurements. \citep{wiki-moving-average}\\

In the city, it is very hard to accurately predict signal loss, even with precise heat maps. This is partly because phones and their antennas vary in quality and partly because signal loss in cities is sporadic and the signal usually returns quickly. Therefore signal loss is not predicted in the city, but only outside. 

%IMPLEMENTATION
\chapter{Implementation}
\label{chap:implementation}
\subsubsection{Dispatcher}
\label{sec:dispImplementation}
The Dispatcher has been implemented as conditional checking for methods received from the client in XML-format. It will statically look for the methods that exist in the Admin class, and call the right method accordingly. This means that the Admin will not be able to change the functionality without having to rework the Dispatcher. The Dispatcher searches for a method call by checking if the XML contains a method name known to be located in the Admin class. The Dispatcher handles all calls to the Game Thread by looking for an XML-tag called \textit{<GameId>}. Throughout the framework, this XML-tag tells what game the call belongs to. Each game is uniquely identified by an Id. By knowing this, we can conclude that whenever the Dispatcher receives the \textit{GameId}-tag, the call should be dispatched to a Game Thread. We then need to figure out from the XML-text inside this tag to target the correct game thread on the server. The Dispatcher then dynamically invokes the correct method on the correct game thread specified by this XML-text. The code-implementation for the dynamic method call can be seen in listing \ref{lst:dispdyncall}.

\begin{lstlisting}[caption={Dynamically invoking methods on game threads}, language=C, label={lst:dispdyncall}]
//Locate method call in XML sent from client
string methodCall = xh.GetMethodCallFromXML(xml);
object[] methodParams = {xml};

//Invoke method on correct game thread
Type type = typeof(GameThread);
MethodInfo method = type.GetMethod(methodCall);
GameThread c = AsynchronousSocketListener.gameThreadPool.GetGameInstance(xh.GetGameIdFromXML(xml));
string result = (string)method.Invoke(c, methodParams);
\end{lstlisting}

The Dispatcher will always return a XML-string to the Async IO.
\subsection{Game Thread}
\label{sec:gamethreadimpl}
Each active game is running as an instance of the game thread class on a separate thread on the server. The thread will not perform any actions on its own, but sits idle until another part of the server initializes a call to it. The parts of the server which can perform a call to a game thread are Admin, Dispatcher, GlobalTimerThread. Further explanation can be seen in design of game thread\ref{designgamethread}. 

The game thread has to handle a lot of different calls from different parts of the server, however it does not hold any game specific information itself. Every change to a game state, locations, events, actions, etc, will change the state of a game in the database. Therefore the game thread acts as a connection between information stored in the database and the requests made by the client.

All public methods in the game thread receives a string as parameter, because the methods within game thread handles the XML by itself. This XML is made by the client and contains all parameters the method needs. Each method in the game thread can then use the XMLhandler to pull out the desired data and convert it to the right type. The method hereby receives an object as the right type which it can proceed to work with.

The game thread contains private methods for handling private calculations. An example of this could be the method called \textit{PlaceObjectsOnRectangularBoard}, which is called when creating a game. It takes an argument for the number of objects to place on the board, the south-east boundary coordinate, the north-west boundary coordinate, and the collision radius to keep between the objects to handle the spread on the board. 



\section{Implementation of XML handler}
\label{sec:xmlhandlerimpl}


%CONCLUSION
\section{Conclusion}
Our problem statement asks the question: \textit{How can a flexible framework for creating multiplayer games be developed?} In this project we show a way to develop such a framework. We decided to divide this framework into 3 layers called Dispatcher Service, Logic Layer and Data Layer. This way we could develop and encapsulate each module in the framework and connect each module through a communication interface. We are satisfied with our division and think it fulfills the desired encapsulation and structure we wanted. 

We are satisfied with our network communication in the Dispatcher Service, and it fulfills the needs of our project. We would like the Dispatcher itself to be extend further and possibly be more generic. As it is now, it only contains the static methods we provided it with. These methods are a limited amount and heavily inspired by the game we developed alongside our framework. 

The Event Timer in the Logic Layer is satisfactory and implemented with flexibility as main priority. As a module, the Event Timer is very simple but the task it fulfills is significant. The Admin is implemented similar to the Dispatcher and can be extended further and be made more generic. It fulfills the needs of our game and the shortcomings can possibly be solved by forcing the user to write more code in the Game Thread class, to make up for these.
XML-handler and XML-builder suffer from same shortcomings as the admin and dispatcher, and have the same solution.

The Data Layer implements a very satisfying interface to the database and the database complexity, and has all the functionality we desired. The interface between Logic Layer and Data Layer is well structured. It provides a good abstraction for the data storage, along with providing a collection of methods for fetching data. The collection could be extended and made more comprehensive, it suffers from the same shortcomings as the modules in the Logic Layer. However, the Data Layer proves a successful design and would be completely satisfying for the framework, once it was extended to offer all database calls. 

Altogether the implementation of the framework proves that the design is successful in showing how a flexible framework can be developed. The extensiveness of the framework could be improved but the modules form the desired functionality that allows a user to create location-based multiplayer games. 


\bibliographystyle{chicago}
\bibliography{litteratur/litteratur}

% Appendix
\appendix

\label{LastPage}
\end{document}