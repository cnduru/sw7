\thispagestyle{empty}
\section*{Preface}
This report was written by six 7th semester software engineering students from Aalborg University.\\

...

\noindent The source code for this project is accessible through one of the following links:
\begin{itemize}
\item \url{http...}
\end{itemize}

\vspace{.2cm}
\noindent Thanks here

\begin{table}[H]
	\centering
	\vspace{2cm}
		\begin{tabular}{c c c}
			\underline{\phantom{JAERJAERJAERJAERGO}} & \phantom{cookies} & \underline{\phantom{JAERJAERJAERJAERGO}} \\
			Casper Holst Laustsen & \phantom{cookies} & Christoffer Ndũrũ\\[1.5cm]
		    \underline{\phantom{JAERJAERJAERJAERGO}} & \phantom{cookies} & \underline{\phantom{JAERJAERJAERJAERGO}} \\
			Dan Skøtt Petersen & \phantom{cookies} & Johan Leth Gregersen\\[1.5cm]
			\underline{\phantom{JAERJAERJAERJAERGO}} & \phantom{cookies} & \underline{\phantom{JAERJAERJAERJAERGO}} \\ 
			Kristian Mikkel Thomsen & \phantom{cookies} & Morten Møller Jakobsen\\[.5cm]				
		\end{tabular}
\end{table}


%
%\thispagestyle{empty}
%\section*{Preface}
%This report was written by four 6th semester software engineering students from Aalborg University.\\
%
%\noindent The report details the work the group did to produce two Android apps (Life Stories and Week Schedule) for use in the Graphical Interface Resources for Autistic Folk (GIRAF) framework. The language of the apps is Danish by request of the customers, however, throughout this report they are referenced by their English names. Beyond describing the work done by the group, the report also describes the work performed in collaboration with other groups.\\
%
%\noindent The source code for this project is accessible through one of the following links:
%\begin{itemize}
%\item \url{https://www.dropbox.com/s/od3vdobckuhtq9a/tortoise.zip}
%\item \url{http://goo.gl/CHRxEp}
%\end{itemize}
%
%\vspace{.2cm}
%\noindent We would like to extend thanks to our supervisor, Xike Xie, for providing guidance throughout the project, and to the involved customers from the institutions for their feedback and ideas. We would also like to thank Ulrik Nyman for starting the GIRAF project.
%%Tak til Xike, Ulrik og institutioner (specielt test-subject)?
%
