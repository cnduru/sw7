\section{Location Detection and Predicting Signal Loss}
\subsubsection{Determining Location}

When deciding what web-content is required available it is important to know the locational context of the user, specifically whether the user is in a urban area or not. Different methods can be used to determine this, such as basing it on map data or connectivity.

When basing it on map data the idea is to compare the current GPS-coordinates to the map data, in order to determine 
\subsection{Predicting Signal Loss}
To predict when a device might suffer a signal loss and become disconnected from the internet, two different scenarios must be considered. The scenarios are as follows\\

\begin{itemize}
\item The device is moving into the country side
\item The device is in the city\fxwarning{should this be removed? is the scenario not needed anymore?}
\end{itemize}

In order to predict when the signal may be at risk of being lost in the first scenario, it makes sense to look at measurements from the antenna of the device. A number of measurements, five for example, can then be stored on the device. These values should be error corrected (e.g. using standard deviation) to make sure sudden fluctuations will not trigger an alarm, telling the device that it has lost its signal. Since coverage is usually worse outside of big cities compared to in the cities, the signal will be lost gradually and not suddenly. Therefore it makes sense to measure a number of antenna measurements, and decide whether they are falling at a rate which would indicate heading towards the country side.\\

This functionality uses the moving average, which has the purpose of smoothing out temporary fluctuations in measurements. \citep{wiki-moving-average}\\

In the city, it is very hard to accurately predict signal loss, even with precise heat maps. This is partly because phones and their antennas vary in quality and partly because signal loss in cities is sporadic and the signal usually returns quickly. Therefore signal loss is not predicted in the city, but only outside. 