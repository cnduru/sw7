\section{Conclusion}
Our problem statement asks the question: \textit{How can a flexible framework for creating multiplayer games be developed?} In this project we show a way to develop such a framework. We decided to divide this framework into 3 layers called Dispatcher Service, Logic Layer and Data Layer. This way we could develop and encapsulate each module in the framework and connect each module through a communication interface. We are satisfied with our division and think it fulfills the desired encapsulation and structure we wanted. 

We are satisfied with our network communication in the Dispatcher Service, and it fulfills the needs of our project. We would like the Dispatcher itself to be extend further and possibly be more generic. As it is now, it only contains the static methods we provided it with. These methods are a limited amount and heavily inspired by the game we developed alongside our framework. 

The Event Timer in the Logic Layer is satisfactory and implemented with flexibility as main priority. As a module, the Event Timer is very simple but the task it fulfills is significant. The Admin is implemented similar to the Dispatcher and can be extended further and be made more generic. It fulfills the needs of our game and the shortcomings can possibly be solved by forcing the user to write more code in the Game Thread class, to make up for these.
XML-handler and XML-builder suffer from same shortcomings as the admin and dispatcher, and have the same solution.

The Data Layer implements a very satisfying interface to the database and the database complexity, and has all the functionality we desired. The interface between Logic Layer and Data Layer is well structured. It provides a good abstraction for the data storage, along with providing a collection of methods for fetching data. The collection could be extended and made more comprehensive, it suffers from the same shortcomings as the modules in the Logic Layer. However, the Data Layer proves a successful design and would be completely satisfying for the framework, once it was extended to offer all database calls. 

Altogether the implementation of the framework proves that the design is successful in showing how a flexible framework can be developed. The extensiveness of the framework could be improved but the modules form the desired functionality that allows a user to create location-based multiplayer games. 