\subsection{User Friendly}
We wanted to make our framework user friendly. This was to ensure the ease of use and therefore more tempting for users to learn. One of the decision we chose to ensure the user friendliness was to only ask the user to code in the Game Thread on the server. This means that the users can make use of the framework interfaces and program the desired game with all the logical computations placed only in the Game Thread. This requires the user to learn the interfaces between the Game Thread and the rest of the framework. This is where our user friendliness could be improved upon, since we have a few vague method names. We also demand the user to learn a few requirements to use the framework properly. These are requirements like the name consistency between the Client XML-encoding and the method-names used in the Game Thread. To make use of a method in Game Thread, the Client has to use the exact name in the first XML-tag. 

The user also has to develop a Client that can communicate with the developed game server and display the GUI for the player. The Client will have to make use of framework functionality like logging in by building the right call to the server. This cannot be programmed in the Game Thread, because it is part of the Admin module.  

We could make improvements on the user friendliness of the framework to ensure a better name-convention. Additionally, we could reduce the requirements and make the framework more dynamic by not demanding specific name-conventions. Lastly, the biggest thing we could improve is our error-checking and error-handling that always returns a default error-message stating that the message was not understood when receiving unknown XML-tag. Additionally, error-handling for methods being provided with wrong parameters do not return an error message and often throws an exception. This is a big reason why we are not satisfied with our user friendliness.