\chapter{Introduction}
\label{chap:intro}
%http://thesistips.wordpress.com/2012/03/25/how-to-write-your-introduction-abstract-and-summary/

% It introduces the problem and motivation for the study.

% - Tell the reader what the topic of the report is.
% - Explain why this topic is important or relevant.

% It provides a brief summary of previous engineering and/or scientific work on the topic.
% - Here you present an overview what is known about the problem.  You would typically cite earlier studies conducted on the same topic and/or at this same site, and in doing so, you should reveal the yawning void in the knowledge that your brilliant research will fill.

% It outlines the purpose and specific objectives of the project.
Development of an online multiplayer game serving many users can be split into two parts: Creating the actual game functionality, and creating the underlying architecture to support it. When developing new games these parts are possibly mixed together as the underlying architecture is only used for the specific game. This has the consequence that the developer possibly writes a game specific architecture, and not a general architecture. In some cases, the developer may want to create other similar games with the same underlying structure, but since the architecture and the game itself are intertwined, it is difficult to reuse the same structure. The developer would have to first search through the previous model and extract the desired features, which costs some time. Furthermore, the functionality might not be an exact match to the needs of the new game, requiring additional time to adjust the code.

The focus of this project is to create a flexible and reliable architecture for a server that supports client based multiplayer games. The goal is to allow easy development of multiplayer games for smartphones, and to make it possible for developers to focus on the game development without having to worry about managing the underlying architecture.

The framework will focus on creating a platform for games centered around the location of the player. It has to be scalable, and therefore the server should easily be able to handle multiple games and multiple devices (players) at the same time. This means it is a requirement that multiple clients should be able to connect to the server and perform different actions at the same time. The framework should be able to handle GPS-coordinates, using it as the base of some sort of action, depending on the game implementation of the game, e.g. in a shooter game you are able to shoot players within a certain distance. 

Creating new games from scratch requires both time and monetary investments. While some companies might have these resources available, it could increase the potential profit if some of these resources could be saved. The extra time either means a bigger profit for the developer or more time to work on game features. We argue that creating a flexible framework could help improve this situation.

A framework serves as a basic set of functionality and structural guidelines for the game to be built upon. Overall, game development rarely fits traditional software development process such as the waterfall method, as requirements can change quickly\fixme{source?}. Fairly, an agile development process is often used \cite{Gamedevelopment}. The result is that functionality might have to change on a frequent basis. Whether this means new functionality or rewriting already existing functions - a game with an intertwined framework and game will take more time to rewrite, as components often have or provide dependencies for other parts.\\

Consider a third person shooter game where functionality revolves around specific gun types and their functionalities. Far into the development, another gun type might be a new requirement. The developer would, depending on the existing framework have to rewrite much of the code regarding guns in general to suddenly include new functionality. This requires a lot of extra time compared to having a framework that allows easy addition and removal of gun functionality. Worse is examples where you have to abandon an idea and create something entirely different. Having to rewrite all the code would be tedious, especially when it involves the framework. Having an independent framework could help alleviate this issue.

The size and the amount of divisions of a developing company also matters. The more different people working on it, the more precision is needed both in the documentation and the framework. Having multiple people writing code on the same game requires the underlying architecture to be as consistent as possible. Changing something in the framework or the underlying architecture requires a notification to all different divisions and people working on the game. It might result in some code refactoring \cite{Gameprod}. These companies often create their own framework and document it through the game design document \cite{Gamedesign} in order for it to fit exactly their needs. This causes them to have a consistent and steady framework to work on, and work individually without consulting different divisions. Smaller developer companies might not have the luxury of creating a completely game specific framework for each of their projects. This results in smaller developer companies often having to spend more work hours adjusting their code to what their co-developers have coded.

Another problem with developing different games from scratch as opposed to having a standardized framework is compatibility. Having different types of devices with different versions of various operating systems means that some devices might not be compatible with the program. What this means is that programs act differently on different devices. With an existing framework this only has to be implemented once. Having a framework that is already compatible with the different devices allows the coder to focus solely on the game, and therefore limiting the amount of time needed for creating portable code.

Constructing a framework that could be used by different developers could save the developer a lot of time and money, if the developer uses the framework as intended. It could and should be easier to use for the developer - giving him less work hours needed to create the same content he otherwise would. The struggle with creating a framework like this is that it has to be flexible enough for game developers to create diverse games, but still be precise enough for the developers to not have to write an a lot of code, in order to fit it to the game.

The goal is widely pursued in the form of game engines and general APIs. A large amount of games being developed, are developed on top of already existing game engines. An example of a game engine is the `Adventure Game Studio', providing an API for creating games in third person perspective along with support for pre-rendering. The developer, Chris Jones, has created different functionality he deemed important for creating these games and published it in the form of a simple IDE for game developers \cite{adv-game}. It is specific in the way it focuses on third person adventure games, making specific functionality.
Another example of game engines, is full blown engines with multiple purposes for a specific genre. One of the most used engines for first person shooters is the "Unreal Engine" which is used for many major shooter or fighter games. It is used both as a visual aid, providing 3D rendering but also provides pre-rendering functionality.

% It provides a ‘road map’ for the rest of the report.
% - This is so that the reader knows what’s coming and sees the logic of your organization.
% - Describe (in approximately one sentence each) the contents of each of the report/thesis chapters.

\section{Problem Statement}
\label{sec:probstate}
When creating games time and money are valuable resources. To ensure a high amount of quality code, the developer has to spent a lot of time writing code that could be found in a generic framework, increasing the cost unnecessarily. This naturally raises the question:

%\mbox{\parbox[b][2.5cm][c]{0.95\textwidth}{\textit{How can we develop a framework for creating multiplayer games - increasing the experience for the developers.}}}

\mbox{\parbox[b][2.5cm][c]{0.95\textwidth}{\textit{How can a flexible framework for creating multiplayer games be developed?}}}\fxfatal{Bør stadig overvejes!}

\noindent The idea is simple: we want to create a framework for creating simple location-based multiplayer shooter games. In order to properly answer the stated problem, these goals should be fulfilled:

\paragraph{Flexibility} The framework has to be as flexible as possible. I.e., it should allow implementation of as broad a spectrum of games as possible within the mentioned genre.
\paragraph{Scalability} The framework should be able to handle multiple games, each with multiple users. The purpose of hosting several games at once is to allow development of relatively short games with a limited number of players each, e.g. a group of friends.
\paragraph{Ease of Use} The framework has to be easy to use for developers. If it is too complicated it will not be as attractive to start using - the idea is to make development easier rather than harder.





%Scenarios
%\section{Scenario 1: Maps \& Navigation - City Areas}
In a large city, for example one of the 5 largest cities in Denmark, data connectivity can be obscured by many big buildings, concrete parking garages or when entering the the subway. This type of data connection loss can be unpredictable and happen in an instant, but it can be expected to be relatively short. 

A scenario can be a travelling salesman arriving at an unknown large city. The salesman is trying to find a customer and have to navigate through public transportation. The customer is awaiting the salesman for a business meeting, and therefore the salesman is limited by time. The salesman is relying on information about public transport timetables and a journey planner to direct him to the customer. 

A number of challenges may arise for the salesman regarding connectivity issues. The salesman may have plans to ride one bus to the customer, but upon arriving at the bus stop he realises the bus is delayed. He now has to consult his device for an alternative bus to ride. Since the salesman is under time pressure, he is relying heavily on having data-connection. Another challenge may arise if the salesman descents into the metro-station. The salesman may have become confused about which train to enter, and is now relying on still having a data connection. The subway is underground and built from concrete, which may severely reduce the strength of the data connection. 

The challenges arise when the salesman is under time pressure and has to deal with unpredictable traffic and public transportation. The salesman would be more secure if a service would be able to provide the desired information even if the device has lost data-connection.
%\section{Scenario 2: Maps \& Navigation - Countryside Areas}
When travelling out of cities, data connectivity can be unavailable for a longer period of time. This type of data connection loss is slowly fading away as you move further away from the connection source.

A scenario can be a couple going for a walk in the woods on the countryside. They are visiting an unknown area and walks into the woods. They happen to get lost and wants to find their way back to the car, but in the woods no data connection is available and they are lost.

A modification of this scenario could be a couple that wants to get additional knowledge of the area they are walking around in, e.g. visiting Rold skov in Denmark, they might want to read the stories about the robbers from Rold. It is also possible that they want to know about the attractions in the area, and directions to them.

The challenges are predicting that the data connection will be lost in the near future, and what services should be attempted to download to a mobile device. 

%\section{Scenario 3: Bus and Train}
In this scenario the problem be solved is to minimize Internet drop-outs when travelling on frequently used bus or train lines. The idea is to collect data on a user base and use that data to identify areas where the user might lose connection.

The solution should be able to detect travelling by bus or train and predict the route and destination. It should reduce the user experience of no Internet during the travel and make relevant information available for the destination, should it be in a poorly connected area.



%In order to do so several approaches are available.

%A simple approach is to measure the signal strength of the device and use that as the base for predicting when connection will be unavailable. Using this method a stable but slowly dropping signal strength might indicate a future with no connection available, in this event the required web resources should be cached so they are available doing an offline period. Another scenario could when the signal is unstable but the average remains high, this could indicate interference in the area but no risk of a significant offline period, in this case caching is not necessary.
%
%Another approach is to base the prediction on historical data of the near area. This can be done by gathering data of where Internet connections usually drops and build a map for the available providers. Then the caching will be done when a device is predicted to move into such an area.

%Problem statement
% How can a user be guaranteed to always have relevant information available, regardless of network connectivity?
\section{Chapter Descriptions}
The following is a brief description of each chapter in the report

% add more here as report progresses %
\paragraph{From Idea to Framework}
This chapter will give a general idea behind the choices made - why we chose to make a game and what outcome expect to get from it. It will give a short description of the overlying architecture and goal of the framework. 
\paragraph{The Game}
This chapter will describe the game we used a foundation for creating a framework. It will describe the client, how it was designed and lastly how it was implemented.
\paragraph{The Framework}
This chapter describes all of the choices considered for the framework. It will describe how the game influenced the server and framework and advantages and disadvantages of this. Furthermore it will describe design and implementation of the server and database. Lastly a section will describe tests and the result of it.
\paragraph{Conclusion}
A brief summary of the entirety of the project and an evaluation of whether the project upholds the goals of the project.

