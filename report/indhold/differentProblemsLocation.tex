\section{Different Needs in Different Contexts}\label{sec:context}
In order to solve the problems described in the scenarios it becomes important for a single application to tell the different contexts apart, so it is able to solve the right problem at the right time. This is necessary because not all of the problems are relevant in all contexts, e.g. predicting when a connection might drop with the intention of downloading a map would not have much use if the user were traveling by bus.

The context of a user can be split into two categories: the locational and the activity context. The locational context refers to the location of the user, e.g. a user in a forest is likely to have different needs than a use in a city or at home. The activity context refers to the activity of a user, e.g. whether the user is traveling in a vehicle or by foot, or whether or not they have an appointment. The application must be able to determine the user context in order to adhere to the user's needs.