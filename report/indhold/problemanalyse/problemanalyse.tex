\chapter{Problem Analysis}
\label{chap:prob}
When using a mobile device on the go, a good network connection is not always available, but relevant and updated data may still be needed. This naturally raises the question:

\textit{How can a user be guaranteed to always have relevant information available, regardless of network connectivity?}\\

\noindent In the following, we will present some possible scenarios, in which this problem can occur.

%Scenarios
%\section{Scenario 1: Maps \& Navigation - City Areas}
In a large city, for example one of the 5 largest cities in Denmark, data connectivity can be obscured by many big buildings, concrete parking garages or when entering the the subway. This type of data connection loss can be unpredictable and happen in an instant, but it can be expected to be relatively short. 

A scenario can be a travelling salesman arriving at an unknown large city. The salesman is trying to find a customer and have to navigate through public transportation. The customer is awaiting the salesman for a business meeting, and therefore the salesman is limited by time. The salesman is relying on information about public transport timetables and a journey planner to direct him to the customer. 

A number of challenges may arise for the salesman regarding connectivity issues. The salesman may have plans to ride one bus to the customer, but upon arriving at the bus stop he realises the bus is delayed. He now has to consult his device for an alternative bus to ride. Since the salesman is under time pressure, he is relying heavily on having data-connection. Another challenge may arise if the salesman descents into the metro-station. The salesman may have become confused about which train to enter, and is now relying on still having a data connection. The subway is underground and built from concrete, which may severely reduce the strength of the data connection. 

The challenges arise when the salesman is under time pressure and has to deal with unpredictable traffic and public transportation. The salesman would be more secure if a service would be able to provide the desired information even if the device has lost data-connection.
%\section{Scenario 2: Maps \& Navigation - Countryside Areas}
When travelling out of cities, data connectivity can be unavailable for a longer period of time. This type of data connection loss is slowly fading away as you move further away from the connection source.

A scenario can be a couple going for a walk in the woods on the countryside. They are visiting an unknown area and walks into the woods. They happen to get lost and wants to find their way back to the car, but in the woods no data connection is available and they are lost.

A modification of this scenario could be a couple that wants to get additional knowledge of the area they are walking around in, e.g. visiting Rold skov in Denmark, they might want to read the stories about the robbers from Rold. It is also possible that they want to know about the attractions in the area, and directions to them.

The challenges are predicting that the data connection will be lost in the near future, and what services should be attempted to download to a mobile device. 

%\section{Scenario 3: Bus and Train}
In this scenario the problem be solved is to minimize Internet drop-outs when travelling on frequently used bus or train lines. The idea is to collect data on a user base and use that data to identify areas where the user might lose connection.

The solution should be able to detect travelling by bus or train and predict the route and destination. It should reduce the user experience of no Internet during the travel and make relevant information available for the destination, should it be in a poorly connected area.



%In order to do so several approaches are available.

%A simple approach is to measure the signal strength of the device and use that as the base for predicting when connection will be unavailable. Using this method a stable but slowly dropping signal strength might indicate a future with no connection available, in this event the required web resources should be cached so they are available doing an offline period. Another scenario could when the signal is unstable but the average remains high, this could indicate interference in the area but no risk of a significant offline period, in this case caching is not necessary.
%
%Another approach is to base the prediction on historical data of the near area. This can be done by gathering data of where Internet connections usually drops and build a map for the available providers. Then the caching will be done when a device is predicted to move into such an area.

%Problem statement
% How can a user be guaranteed to always have relevant information available, regardless of network connectivity?