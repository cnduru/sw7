\section{Scenario 1: Maps \& Navigation - City Areas}
In a large city, for example one of the 5 largest cities in Denmark, data connectivity can be obscured by many big buildings, concrete parking garages or when descenting into the subway. This type of data connection loss can be unpredictable and happen in an instant. 

A scenario can be a travelling salesman arriving at an unknown large city. The salesman is trying to find a customer and have to navigate through public transportation. The customer is awaiting the salesman for a business meeting, and therefore the salesman is limited by time. The salesman is relying on information about public transport timetables and a journey planner to direct him to the customer. 

A number of challenges may arise for the salesman regarding connectivity issues. The salesman may have plans to ride one bus to the customer, but upon arriving at the bus stop he realises the bus is delayed. He now has to consult his device for an alternative bus to ride. Since the salesman is under time pressure, he is relying heavily on having data-connection. Another challenge may arise if the salesman descents into the metro-station. The salesman may have become unwary about which train to enter, and is now relying on still having data-connection. The subway is underground and build from concrete which may severely reduce the data-connection. 

The challenges arise when the salesman is under time pressure and has to deal with unpredictable traffic and public transportation. The salesman would be more secure if a service would be able to provide the desired information even if the device has lost data-connection.