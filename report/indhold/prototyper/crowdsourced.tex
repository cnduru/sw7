\section{Crowdsourced analysis}
\label{sec:crwdsourc}
Based on nearby devices network-connection - this trigger solution will track where devices previously lost connection and where they regain connection. Use this information to make a live registration of nearby connection dead zones, with a limited lifespan. Since new connectivity-data will be available more often than planned dead zones scans, the lifespan can be kept low and the available data may be kept dynamic for highly populated areas. The lifespan of connectivity-data may then be increased in areas less populated.

According to [link], crowdsourcing can be used to outsource routine tasks to a large unknown crowd of participants. The participants require no or limited skills to participate in the dead zone analysis, they will most likely not even notice, since their participation will be automated after installation of an mobile-application. This mobile application will automatically log data and send it to a server. Uniting this data, from a large number of participants, will give a live map of connection dead zones. 

(temp link until JabRef is set up: http://raptor1.bizlab.mtsu.edu/s-drive/DMORRELL/Mgmt\%204990/Crowdsourcing/Schenk\%20and\%20Guittard.pdf)


\begin{table} [h]
   \begin{center}
   \begin{minipage}{\textwidth}
      \centering
      \begin{tabularx} {\textwidth} { X | X  }
         \hline
		 & \\
         Advantages & Disadvantages \\
		& \\\hline
		& \\
         \tabitem Connectivity-data will have short lifespan and refreshed often & \tabitem Requires a large number of participants to effectively work \\
         \tabitem Does not require field analysis by designer & \tabitem Requires participants to resident over a wide area \\
		& \\\hline
      \end{tabularx}
      \caption{Crowsourced analysis advantages and disadvantages}
      \label{tab:crowdana_adv}
   \end{minipage}
   \end{center}
\end{table}
