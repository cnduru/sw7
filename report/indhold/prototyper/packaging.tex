\subsection{Packaging}
\label{sec:packingadv}

A solution to losing network services and therefore the possibility to download maps, are to detect when the device is losing network-connection in appropriate time. Hereby the device may start downloading maps over the nearby area and store for use once the network connection is lost. These maps should then be compartmentalized into groups with different prioritzation, based on a criteria of nearest-package-first. 

Downloading the by nearest-package-first requires an indexation of map-packages for a given location of a device. Since our service are to provide maps even when network connection is lost, we intend to use a map-service like Google Maps and not develop our own. Google Maps already divided a map into small packages, which we could use for downloading and displaying upon request depending on simple GPS-coordinates. New packages will be downloaded and displayed when the user moves the map on a mobile-device. The problem that has to be dealt with, are which packages to download without knowing more than GPS-coordinates. 

We put forward three solutions to this problem:
\begin{itemize}
\item Define a circular perimeter around the GPS-coordinates. Start downloading map-packages on the device GPS-coordinates, then prioritize after nearest-package-first until the perimeter is covered with downloaded packages.
\item Define a rectangular perimeter around the GPS-coordinates and proceed as above.
\item Process the GPS-coordinates over time, which will provide direction and velocity of the device. This should be used to download packages in the direction that the device is moving, and disregard the package behind the device. Considering the velocity, the width of required packages can be defined and as velocity increase packages further away can be downloaded. This solution does not follow nearest-package-first defined on first set of GPS-coordinates but rather try to dynamically follow nearest-package-first based on time and movement.
\end{itemize}




\begin{table} [h]
   \begin{center}
   \begin{minipage}{\textwidth}
      \centering
      \begin{tabularx} {\textwidth} { X | X  }
         \hline
		 & \\
         Advantages & Disadvantages \\
		& \\\hline
		& \\
         \tabitem adv & \tabitem disadv\\
         \tabitem adv & \tabitem disadv \\
		& \\\hline
      \end{tabularx}
      \caption{this is a dummy}
      \label{tab:packingadv}
   \end{minipage}
   \end{center}
\end{table}
