\section{Separation of Framework and Game}
%It should be possbile to build a new game on the same framework.
The framework and the game should be split into two completely separate components. This enables the creation of different games on the same framework.

%Why is the framework and game separated?
Because of this, developers wanting to create online multiplayer games can reuse the framework without having to touch the source code. This means that there is going to be a clear line between the game and the framework, having no game specific elements in the framework.

%Why is it a good idea, and what are the tradeoffs?
%What would happen if it is not split up?
Developing the framework and game as two completely independent components has a number of advantages. It makes it easier to avoid writing code which is intertwined between the framework and the game.

A clear code-wise distinction between framework and server produces a cleaner, more independent program structure. However if the framework and game are not separate components, it could be easier to develop a single game, since the strict separation between the two can cause a development overhead. This can occur because code has to be planned more carefully when written for the separated components.

%Discuss the constraints of the framework (e.g. database schema, efficiency).
Because the framework and game are going to be separated components, the framework will have some limitations. The database tables have to be fixed, since developers should not need to change the database and framework code. A database change will warrant some code in the framework to be rewritten, as the controller between the database and rest of the framework will be invalid. If the developers have to rewrite code in the framework, in order for their game to work, the point of the framework is lost. It is however possible to create a database which is very flexible, by having very generic entities and relationships, and having many nullable attributes. This will however make the database bigger in terms of memory usage than if it was designed for the specific game.
Having a flexible framework results in loss of efficiency for the developer, both performance- and memory-wise. Being able to fit your game perfectly to your server would mean more efficient code for that single scenario, but never be as flexible as we would want.