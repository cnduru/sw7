\section{Architecture of the Project}
%
% A description of the entire architecture. server, client and which part takes care of what (initially and optimally).
%

% intro
The project as physical entities, is separated into two parts. A server and a client part.

% client
The client acts as a graphical interface for the player, allowing game interaction.
The client does however not perform game specific computations. Rather than changing the state of the game itself, it sends requests to the server, which will compute results based on these. The client then decides what to do with the results. The player could for example interact with someone else in the game - This would make the client send data to the server requesting it to handle this action, notifying the involved players of the result.

% server
The server is where all data regarding players and games are stored. For all games, it keeps track of the current state they are in. It also acts as the center of communication, receiving information based on the actions requested from clients. The server then calculates the result of the action, returning a proper answer to the client and possibly changing the state of a game.

% reasoning
There are multiple reasons for separating the project into server and client entities. One is that it would be power consuming for a client to also act as framework. The mobile phone would have to perform all computations for a hosted game, including operations on a local database. When a central server is used for these computations, it can provide a stable offload for all the clients. Removing the server would add complexity to hosting and joining games. Having a server allows for one static address to connect to, while players otherwise would have to know the address of a specific device hosting a game in order to join it.
Portability is also improved by having most operations performed on the server. This results in the client being thinner, meaning less work involved in porting the client to other platforms.