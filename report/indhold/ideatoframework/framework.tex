\section{Frameworks}

In the context of software, a framework is a solution which provides a set of functionality much like a library, but with small differences. It provides functionality for development of products, either independently or as part of a larger platform.
Generally the flow of control is dictated by the framework rather than the user. The framework often have a default behavior, in case an important functionality is not overridden by the user. The most important detail is however that the framework code should never be modifiable by the user - the user should rather build upon the framework \cite{framework}.

A framework can include other smaller parts such as other programs, tools, APIs or other libraries. A framework can be confused with an API or a library. A framework is a group of classes, interfaces and other pre-compiled code, while APIs are the public front of a framework, and libraries are specifically made for a single purpose. A well-made framework hides most of the code in assemblies and DLL-files and only shows the above mentioned classes and interfaces. An example of a framework for an online store could be responsible for handling the general flow. In this case it would be handling tasks such as `purchase item', `checkout' or `browse warehouse'.

APIs are made for interaction between applications and with a specific purpose e.g. Google Maps API \cite{google-maps-api}. Google Maps is built with an underlying predefined set of functions, and the API is the manual the developer can use to work with Maps. It can be described as the information necessary to use the provided classes and sets of classes.

A library can be described as a very specific framework. It is a collection of implementations with a well-defined interface - a strict set of functions for a specific purpose. An example could be the timer library in C\# \cite{net-timer}, which provides a simple timer class with a lot of different properties, methods and events.

As we do not have an existing framework to build an API for, we need to create the underlying functionality. This means we have to create either a framework or a library, but seeing we have to create wide spectrum of functionality, the optimal solution would be to create a framework. We can then move on to build an API on top of this.