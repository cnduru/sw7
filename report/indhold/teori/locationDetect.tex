\subsection{Determining Location}
As pointed out in \Cref{sec:context} it is important to know the user's locational context, to make sure to supply the most relevant data. It is particularly important to determine whether the user is in an urban area or not. This can be done by comparing the user's current GPS coordinates to the map data, to determine if the coordinates are within the boundaries of a city. This method has a number of disadvantages: If it is a small city, there may not be many cell towers nearby, and the problem may be similar to the one encountered in non-urban areas. Conversely, a well connected area might not be within a city border. On top of that it can be hard to determine if a certain GPS coordinate is within a city or not.

A simpler and more dynamic method is to measure the connectivity and number of nearby cell towers to determine how likely it is for a user to keep an internet connection. In most cases, if there are many cell towers nearby it means the user is in or near a city. Using this method it is hard to be completely sure without facing the same problems as with the first method, but it is a straight-forward solution, which allows the development to focus on more import areas of the problem, and which can relatively easily be improved later.