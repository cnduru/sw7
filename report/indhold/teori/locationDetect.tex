\subsection{Determining Location}
\fxwarning{this is not teori}


When deciding which web content is required to be available, it is important to know the locational context of the user, specifically whether the user is in an urban area or not\fxwarning{ref til hvorfor vi har behov for denne feature}. Different methods can be used to determine this, such as basing it on map data or connectivity.

When basing it on map data the idea is to compare the current GPS coordinates to the map data, to determine if the coordinates are within a city's boundaries. Disadvantages of this method is that map data might not correlate with the actual needs for the given location, for example if the user is in a small town out of range of any antennas. Here the problem is more similar to the problems encountered in non-urban areas. Conversely, a well connected area might not be within a city border.

Another method is to measure the connectivity to determine how likely it is for a user to keep an internet connection. This is done by examining the signal strength of the nearby antenna masts and from that determining the likeliness of losing the connection. \fxwarning{disadvantages?}  