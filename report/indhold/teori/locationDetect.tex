\subsection{Determining Location}
\fxwarning{this is not teori}


When deciding what web-content is required available, it is important to know the locational context of the user, specifically whether the user is in a urban area or not\fxwarning{ref til hvorfor vi har behov for denne feature}. Different methods can be used to determine this, such as basing it on map data or connectivity.

When basing it on map data the idea is to compare the current GPS-coordinates to the map data to determine if the coordinates are within a city boundaries. Disadvantages of this method is that map data might not correlate with the actual needs for the given location. Meaning the user might be in a small town not near any antennas were the problem is more similar non urban areas clause, or the other way around a well connected area might not be within a city border.

Another method is to measure the connectivity, how likely is this user to keep a Internet connection. This is done by examining the signal strength of the nearby antennas and from that determining the likeliness of losing the connection. \fxwarning{disadvantages?}  