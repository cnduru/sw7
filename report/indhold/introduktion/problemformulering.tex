\section{Problem Statement}
\label{sec:probstate}
When creating games the amount of time and funds spent are valuable resource. To ensure a high amount of quality code, the developer has to spent a lot of time writing trivial code and it will thrive the cost up. This naturally raises the question:

\mbox{\parbox[b][2.5cm][c]{0.95\textwidth}{\textit{How can we develop a framework for creating multiplayer games - increasing the experience for the developers}}}


\fxfatal{Kan det skærpes og generaliseres? - Ivan Aaen 2014 - 9/12}
The idea is simple, we want to creating framework for creating simple coordinate-based multiplayer shooter games. In order for the problem statement to be fulfilled, we raised some additional goals to the question asked above:

\begin{itemize}
\item The framework has to be as flexible as possible.
\item The framework has to account for multiple instances of games with multiple users.
\item The framework has to be easy to use. 
\end{itemize}


%Scenarios
%\section{Scenario 1: Maps \& Navigation - City areas}
In a large city, e.g. one of the 5 largest cities in Denmark, data connectivity can be obscured by many big buildings or concrete parking garage. This type of data connection loss can be unpredictable and happen in an instant. A scenario can be a travelling salesman arriving at an unknown large city trying to find a customer on a navigation system. The salesman is limited on time and relying completely on the navigation system to find him the customer. The salesman will rely on a system that can detect which buildings will cause connection loss and hereby download maps and navigation in time.
%\section{Scenario 2: Maps \& Navigation - Countryside Areas}
When travelling out of cities, data connectivity can be unavailable for a longer period of time. This type of data connection loss is slowly fading away as you move further away from the connection source.

A scenario can be a couple going for a walk in the woods on the countryside. They are visiting an unknown area and walks into the woods. They happen to get lost and wants to find their way back to the car, but in the woods no data connection is available and they are lost.

A modification of this scenario could be a couple that wants to get additional knowledge of the area they are walking around in, e.g. visiting Rold skov in Denmark, they might want to read the stories about the robbers from Rold. It is also possible that they want to know about the attractions in the area, and directions to them.

The challenges are predicting that the data connection will be lost in the near future, and what services should be attempted to download to a mobile device. 

%\section{Scenario 3 Bus and Train}
In this scenario the problem be solved is to minimize Internet dropouts when traveling on frequently used bus or train lines. The idea is to collect data on a user basic and use that data to identify areas where the user might lose connection.

The solution should be able to detect traveling by bus or train and predict the route and destination. It should reduce the user experience of no Internet doing the travel and make relevant information available for the destination should it be in a poorly connected area.



%In order to do so several approaches are available.

%A simple approach is to measure the signal strength of the device and use that as the base for predicting when connection will be unavailable. Using this method a stable but slowly dropping signal strength might indicate a future with no connection available, in this event the required web resources should be cached so they are available doing an offline period. Another scenario could when the signal is unstable but the average remains high, this could indicate interference in the area but no risk of a significant offline period, in this case caching is not necessary.
%
%Another approach is to base the prediction on historical data of the near area. This can be done by gathering data of where Internet connections usually drops and build a map for the available providers. Then the caching will be done when a device is predicted to move into such an area.

%Problem statement
% How can a user be guaranteed to always have relevant information available, regardless of network connectivity?