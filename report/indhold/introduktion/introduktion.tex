\chapter{Introduction}
\label{chap:intro}
%http://thesistips.wordpress.com/2012/03/25/how-to-write-your-introduction-abstract-and-summary/

% It introduces the problem and motivation for the study.

% - Tell the reader what the topic of the report is.
% - Explain why this topic is important or relevant.

% It provides a brief summary of previous engineering and/or scientific work on the topic.
% - Here you present an overview what is known about the problem.  You would typically cite earlier studies conducted on the same topic and/or at this same site, and in doing so, you should reveal the yawning void in the knowledge that your brilliant research will fill.

% It outlines the purpose and specific objectives of the project.

Creating new games requires a lot of time and funds. Creating games similar in functionality to games that already exist require the same amount of time and funds as creating the original. Having to create n games that have similar functionality, requires n times as much time and funds if we keep following that logic - but creating some underlying structure when creating the first game cuts down the time and funds needed to create similar games.

The main problem for developers of a new game is the issue of having to create good usable framework for their specific needs. A framework is a basic set of functionality and structural guidelines for the game to be built upon. At the start of a project, game developers set out to create a framework that can handle all of the functions that the game need, without really having specified precisely how those functions are implemented. This means that if functions change or an entirely new functionality need comes up, the framework have to be adjusted to the specific need. An example of this could be how the game handles several users, maybe the game have different needs depending on some variating variable. The amount of workload a developer has to put into creating the framework to match exactly his demand is greatly increased. The different requirements and often change in the framework often makes code incomprehensible.

It requires a lot more work and overview from the small game developers to keep the framework completely up to date. Bigger firms almost always require multiple sections to use the same framework\fxwarning{kilde?} - and therefor consistency (in form of little to no change) is a must. More often than not these firms create their own frameworks to design them exactly the way they need it, but smaller firms might have limited funds or time. This means that smaller developers often either skip on creating features they normally would or create sloppy and inefficient code as a side effect.

Creating a joint framework that could be used by different developers could save smaller and new game developers a lot of time, and create a standard that makes coding on top of it easy. 

The subject is widely pursued in the form of game engines. A large amount of games being developed are developed on an already existing game engines. An example of a game engine is the `Adventure Game Studio' and this is one of the biggest frameworks for creating games in third person pre-rendering. The developer `Chris Jones' have created different functionality he deemed important for creating these games and published it in the form of an simple IDE for inexperienced game developers \cite{adv-game}. Another example of game engines is full blown engines with multiple purposes for a specific purpose. One of the most used engines for first person shooters is the `Unreal Engine' - which is used for almost every major shooter game.

% It provides a ‘road map’ for the rest of the report.
% - This is so that the reader knows what’s coming and sees the logic of your organization.
% - Describe (in approximately one sentence each) the contents of each of the report/thesis chapters.

\section{Chapter Description}
The following is a brief description of each chapter in the report

% add more here as report progresses %
\paragraph{Design and Implementation}
The design part of this section will describe the general idea behind the code - what small choices have been made along the way and why those were chosen. The implementation section will describe how the code for the project was then implemented.
\paragraph{Test}

\paragraph{Future Work}
This section will describe the future work of the project. Ideas and unfinished functions will be described in a perfunctory fashion - giving the possibility for people to advance the project in the direction the original authors intended.
\paragraph{Conclusion}
A brief summary of the entirety of the project and an evaluation of whether the project upholds the goals of the project.