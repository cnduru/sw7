\subsubsection{Event Timer}
\label{sec:eventtimerimpl}
The Event Timer is started when the server launches. It runs in its own thread and it enters an infinite loop where it loops over Game Events. For every iteration in the loop, the Event Timer checks all Game Events in the list, whether the current time is past the timestamp for the Game Event has passed. 

All Game Events is assigned a timestamp upon creation. The Event Timer iterates through the list of Game Events, and checks the timestamp by calling the method \textit{GetTriggerTimestamp()} and compares it to the current time. If the timestamp value is less than the current time (in other words, the event trigger threshold has been passed), the Game Event is triggered by calling the \textit{RunGameEvent()} method assigned to each Game Event. 

The Event Timer provides a method called \textit{AddGameEvent()}, which allows addition of new game events to its internal list of Game Events.

After a Game Event has been triggered, it is removed from the list in the Event Timer. This is done using the native \textit{RemoveAt()} method which are a part of lists in C\#.

The Event Timer class is a part of the framework, whereas the Game Event, described in \Cref{subsec:geventImpl} is not.

\subsubsection{Game Event}\label{subsec:geventImpl}
The Game Event is implemented as a ``container'' for information about Game Events. It has a constructor which initializes variables gameId, eventType and triggerTimestamp. The rest of the class simply consists of getters for these variables. The getters are used by the Event Timer, described in \Cref{subsec:eventtimerimpl}. Instances of the Game Event class are game specific, which means that each of the instantiated Game Events belong to a game, and are therefore not part of the framework. The user of the framework can code any desired Game Events into this class.
