\subsubsection{Game Thread}
\label{sec:gamethreadimpl}
Each active game is running as an instance of the Game Thread class on a separate thread on the server. The thread will not perform any actions on its own, but sits idle until another part of the server initializes a call to it. The parts of the server which can perform a call to a game thread are \texttt{Admin}, \texttt{Dispatcher}, and \texttt{GlobalTimerThread}. Further explanation  of how the game thread works can be found in \Cref{designgamethread}. 

The game thread has to handle a lot of different calls from different parts of the server, however it does not hold any game specific information itself. Every change to a game state, locations, events, actions, etc, will change the state of a game in the database. Therefore the game thread acts as a connection between information stored in the database and the requests made by the client.

All public methods in the Game Thread will receive a string as parameter, because the methods within Game Thread handles the XML by itself. This XML is made by the client and contains all parameters the method needs. Each method in the Game Thread can then use the XMLhandler to pull out the desired data and convert it to the right type. That way the method receives an object as the right type which it can proceed to work with.

The Game Thread may contain private methods for handling private calculations. An example of this could be the  method in our game called \textit{PlaceObjectsOnRectangularBoard}, which is called when creating a game. It takes an argument for the number of objects to place on the board, the south-east boundary coordinate, the north-west boundary coordinate, and the collision radius to keep between the objects to handle the spread on the board. It then places the desired number of object within the boundaries.

This is just some of the functionality that we implemented as part of our game. This helped us develop functionality and shape our framework.