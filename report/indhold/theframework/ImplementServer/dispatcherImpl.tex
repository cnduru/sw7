\subsubsection{Dispatcher}
\label{sec:dispImplementation}
The Dispatcher has been implemented as conditional checking for methods received from the client in XML-format. It will statically look for the methods that exist in the Admin class, and call the right method accordingly. This means that the Admin will not be able to change the functionality without having to rework the Dispatcher. The Dispatcher searches for a method call by checking if the XML contains a method name known to be located in the Admin class. The Dispatcher handles all calls to the Game Thread by looking for an XML-tag called \textit{<GameId>}. Throughout the framework, this XML-tag tells what game the call belongs to. Each game is uniquely identified by an Id. By knowing this, we can conclude that whenever the Dispatcher receives the \textit{GameId}-tag, the call should be dispatched to a Game Thread. We then need to figure out from the XML-text inside this tag to target the correct game thread on the server. The Dispatcher then dynamically invokes the correct method on the correct game thread specified by this XML-text. The code-implementation for the dynamic method call can be seen in listing \ref{lst:dispdyncall}.

\begin{lstlisting}[caption={Dynamically invoking methods on game threads}, language=C, label={lst:dispdyncall}]
//Locate method call in XML sent from client
string methodCall = xh.GetMethodCallFromXML(xml);
object[] methodParams = {xml};

//Invoke method on correct game thread
Type type = typeof(GameThread);
MethodInfo method = type.GetMethod(methodCall);
GameThread c = AsynchronousSocketListener.gameThreadPool.GetGameInstance(xh.GetGameIdFromXML(xml));
string result = (string)method.Invoke(c, methodParams);
\end{lstlisting}

The Dispatcher will always return a XML-string to the Async IO.