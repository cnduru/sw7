\subsubsection{XML handler}
\label{sec:xmlhandlerimpl}
To handle incoming requests from clients, a component was needed to read and translate the XML formatted strings that communication is based on. This resulted in the \textit{XMLhandler} class. The class has many methods, but is simple to use. Given any valid string of XML, the XML handler returns an object of correct type given the value stored in the XML. An example is the method \textit{GetGameIdFromXML}, it will extract an integer with a value specified in the XML. The method call will convert this from string to integer and return it. 

The XML handler is placed in the \textit{Logic Layer}. It is possible to argue for this class to be made static. The XML handler read the XML-interface between the client and the server, it is therefore a specific piece of game logic.  Specific pieces of game logic cannot be considered as part of the framework, hereby the XML handler is not.

All methods are designed to handle individual requests. Given that the type of request is now known, the relevant method can now be used to decode and convert values stored within the XML string into readable formats. It can hereafter be used as valid objects of correct type in the code.

The XML handler encapsulates all value extraction from XML. If a given XML-tag cannot be handled by the XML handler, it is easy to create a new method that handles this given part. The XML handler does not perform any logical calculation.