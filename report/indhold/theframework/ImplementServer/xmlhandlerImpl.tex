\subsubsection{XML Handler} \label{sec:xmlhandlerimpl}
To handle incoming requests from clients, a component was needed to read and translate the XML formatted strings that communication is based on. This resulted in the \textit{XMLhandler} class, which we use in the game we created. The class has many methods for converting XML text into usable programming objects. Given any valid string of XML, the XML handler returns an object of correct type given the value stored in the XML. An example is the method \textit{GetGameIdFromXML}, it will extract an integer with a value specified in the XML. The method call will convert this from string to integer and return it. 

The XML handler is placed in the \textit{Logic Layer}. The XML handler read the XML-interface between the client and the server, it is therefore a specific piece of game logic. Specific pieces of game logic cannot be considered as part of the framework, hereby the XML handler is not.

All methods in the handler are designed to handle individual requests. What kind of request is received by the server is determined by looking at the outer XML tag, which name should match a method name. Given that the type of request is then known, the relevant method can now be used to decode and convert values stored within the XML string into readable formats. It can hereafter be used as valid objects of correct type in the code.

The XML handler encapsulates all value extraction from XML. If a given XML-tag cannot be handled by the XML handler, it is easy to create a new method that handles this given part. The XML handler does not perform any logical calculation. We consider it a good choice to encapsulate this kind of functionality in the framework, but it is still rather game specific. We would like this to become more integrated in the framework, and it would require further iterations with developed games to extend the XML handler to contain a wide variety of getters.