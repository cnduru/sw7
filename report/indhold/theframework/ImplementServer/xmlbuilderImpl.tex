\section{Implementation of XML builder}
\label{sec:xmlbuilderimpl}
To construct responds to the incoming requests from clients, a component was needed to construct an XML formatted string. The result is the \textit{XMLbuilder} class. This class has one or two respond-messages per method that can be called on the server. Some has just got a single respond-messages, because the only thing that changes is the parameters. Others have two respond-messages because they can out different results, e.g. a login-request can either be successful or fail. 

The \textit{XMLbuilder} is structured with a stringbuilder in each method. All methods for XML-building are formatted the same way in the code. If a XML-tag contains more information than an attribute, it will be split into a start-tag on one line, and an end-tag on another line. In between can be one or more tags, but if it only contains one attribute it will be a single line containing the start-tag, the attribute-value, and the end-tag. This makes the code easily readable when searching for the XML-formatting for a respond-message. 