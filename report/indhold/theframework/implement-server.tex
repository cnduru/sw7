%Dispatcher service
\subsection{Dispatcher Service}
The Dispatcher Service handles communication with the client through TCP/IP. We chose this protocol because it ensures data are correct upon arrival at the server, it is offered as library in the C\# programming language that we use to program our framework and we have minor experience using it from previous projects. 

Just like TCP/IP is offered in our programming language, there is also a library for Asynchronous IO. 

We wanted the implementation of our Dispatcher Service to be quick because we wanted to work on more specific framework flexibility and functionality. Network communication has been done in many other applications and we did not want to spend a long time on something that can be copied from others and does not differ far from other implementations. 



\subsubsection{Asynchronous IO}
\label{sec:asyncImplementation}
We decided to base our asynchronous IO implementation on \cite{asynh-imlp}. The implementation consist of a few method to handle async OI, these are \texttt{StartListening}, \texttt{AcceptCallback}, \texttt{ReadCallback} and \texttt{Send}. \texttt{StartListening} is responsible for listening for sockets connections and when a connection is established a AsyncCallback(part of the .net threading library) is made with \texttt{AcceptCallback} to process the incoming data. \texttt{AcceptCallback} prepares for receiving data then calls \texttt{ReadCallback}. \texttt{ReadCallback} reads data from the socket until an end of file tag is read at which point the data is parsed on to the dispatcher. Finally \texttt{Send} is used to send the result of the request back to the client.  %http://msdn.microsoft.com/en-us/library/fx6588te%28v=vs.110%29.aspx
\subsection{Dispatcher}
\label{chap:dispImplementation}
The Dispatcher has been implemented as conditional checking for methods. It will statically look for the methods that exist in the Admin class, and call the right method accordingly. The Dispatcher searches for a method call by checking if the XML contains each method name.

The Dispatcher handles all calls to the game thread by looking for the XML-tag \textit{<GameId>}. Throughout the framework, this XML-tag signals that the current clients requests should be handed over to the game thread\fixme{Dette skal skrives meget tydeligere, uklart hvordan det bliver handled}. The Dispatcher then dynamically invokes the correct method on the specified game thread, provided by the \textit{<GameId>} in the XML. 

\begin{lstlisting}[caption={Dynamically invoking methods on game threads}, language=C]
//Locate method call in XML sent from client
string methodCall = xh.GetMethodCallFromXML(xml);
object[] methodParams = {xml};

//Invoke method on correct game thread
Type type = typeof(GameThread);
MethodInfo method = type.GetMethod(methodCall);
GameThread c = AsynchronousSocketListener.gameThreadPool.GetGameInstance(xh.GetGameIdFromXML(xml));
string result = (string)method.Invoke(c, methodParams);
\end{lstlisting}

The Dispatcher will always return a XML-string to the Async IO.


%Logic Layer
\subsection{Logic Layer}
In this section we will describe the implementation of the Logic Layer as the game developed alongside the architecture. The Admin, Event Timer modules and Game Thread are implemented as an \texttt{Admin}, \texttt{EventTimer} and \texttt{GameThread} class respectively. Additionally XML helper classes \texttt{XMLBuilder} and \texttt{XMLHandler} will be covered.

Since we developed a game alongside our framework, we implemented our own version of the Game Thread. We realised that XML would become a big part of the framework, and we would be writing the same code in multiple places if we did not implement a generic way to handle and build the XML communication interface. Additionally, we could handle conversion from strings of characters to the correct type in the XML handler. 

The Event Timer has been implemented as framework functionality and it works as intended, although we did use it in the game we developed. The implementation allows the user to specify any event in the Game Event class and call it upon a Game Thread. 

The framework allows the use of items, which we in our game designed and developed as weapons. However, the user of the framework can consider in whichever way desired. Our game has implemented a Weapon class, which specifies a list of weapons to be used in the game. If you wish to store a harmful shot hitting a player, reducing the health-points of the player, the database offers the use of \textit{Status Effect} to store this for later use. Since the implementation of Items is game specific, we do not explain our implementation of weapons further.

\subsubsection{Admin}
\label{sec:adminimpl}
The Admin class handles requests from the client that can be classified as not game specific. These are requests like:
\begin{itemize}
\item \textit{VerifiyAccount} handles login-requests from the client.
\item \textit{CreateGame} handles requests to create and start a new game.
\end{itemize}

The Admin will handle these requests by querying the database through the Data Layer, more specific the DBController class. The database will then return the desired data that can be used for something like administrative logic for login, the information in this case being the password for a given username. These logical computations will be available in the Admin class, alongside functionality to fetch all active games stored in the database. We made this functionality, because we want to avoid asking all active Game Threads to respond to a client request looking for all active games. This would be bad design considering our desire to make a scalable framework, imagine if thousands of users requests to see the list of active games. 




\subsubsection{Event Timer}
\label{sec:eventtimerimpl}
The Event Timer is started when the server launches. It runs in its own thread and it enters an infinite loop where it loops over Game Events. For every iteration in the loop, the Event Timer checks all Game Events in the list to see whether the current time is past the timestamp for the Game Event. 

All Game Events are assigned a timestamp upon creation. The Event Timer iterates through the list of Game Events, and checks the timestamp by calling the method \textit{GetTriggerTimestamp()} and compares it to the current time. If the timestamp value is less than the current time (in other words, the event trigger threshold has been passed), the Game Event is triggered by calling the \textit{RunGameEvent()} method assigned to each Game Event. 

The Event Timer provides a method called \textit{AddGameEvent()}, which allows addition of new game events to its internal list of Game Events.

After a Game Event has been triggered, it is removed from the list in the Event Timer. This is done using the native \textit{RemoveAt()} method which are a part of lists in C\#.

The Event Timer class is a part of the framework, whereas the Game Event, described in \cref{subsec:geventImpl} is not.

\subsubsection{Game Event}\label{subsec:geventImpl}
The Game Event is implemented as a ``container'' for information about Game Events. It has a constructor which initializes variables gameId, eventType and triggerTimestamp. The rest of the class simply consists of getters for these variables. The getters are used by the Event Timer, described in \cref{subsec:eventtimerimpl}. Instances of the Game Event class are game specific, which means that each of the instantiated Game Events belong to a game, and are therefore not part of the framework. The user of the framework can code any desired Game Events into this class.
\subsection{Implementation of Game Thread}
\label{sec:gamethreadimpl}
Each active game is running as an instance of the game thread class on a separate thread on the server. The thread will not perform any actions on its own, but sits idle until another part of the server initializes a call to it. The parts of the server which can perform a call to a game thread are Admin, Dispatcher, GlobalTimerThread. Further explanation can be seen in design of game thread\ref{designgamethread}. 

The game thread has to handle a lot of different calls from different parts of the server, however it does not hold any game specific information itself. Every change to a game state, locations, events, actions, etc, will change the state of a game in the database. Therefore the game thread acts as a connection between information stored in the database and the requests made by the client.

All public methods in the game thread receives a string as parameter, because the methods within game thread handles the XML by itself. This XML is made by the client and contains all parameters the method needs. Each method in the game thread can then use the XMLhandler to pull out the desired data and convert it to the right type. The method hereby receives an object as the right type which it can proceed to work with.

The game thread contains private methods for handling private calculations. An example of this could be the method called \textit{PlaceObjectsOnRectangularBoard}, which is called when creating a game. It takes an argument for the number of objects to place on the board, the south-east boundary coordinate, the north-west boundary coordinate, and the collision radius to keep between the objects to handle the spread on the board. 



\subsubsection{XML handler}
\label{sec:xmlhandlerimpl}
To handle incoming requests from clients, a component was needed to read and translate the XML formatted strings that communication is based on. This resulted in the \textit{XMLhandler} class. The class has many methods, but is simple to use. Given any valid string of XML, the XML handler returns an object of correct type given the value stored in the XML. An example is the method \textit{GetGameIdFromXML}, it will extract an integer with a value specified in the XML. The method call will convert this from string to integer and return it. 

The XML handler is placed in the \textit{Logic Layer}. It is possible to argue for this class to be made static. The XML handler read the XML-interface between the client and the server, it is therefore a specific piece of game logic.  Specific pieces of game logic cannot be considered as part of the framework, hereby the XML handler is not.

All methods are designed to handle individual requests. Given that the type of request is now known, the relevant method can now be used to decode and convert values stored within the XML string into readable formats. It can hereafter be used as valid objects of correct type in the code.

The XML handler encapsulates all value extraction from XML. If a given XML-tag cannot be handled by the XML handler, it is easy to create a new method that handles this given part. The XML handler does not perform any logical calculation.
\subsubsection{XML Builder}
\label{sec:xmlbuilderimpl}
To construct responds to the incoming requests from clients, a component was needed to construct an XML formatted string. The result is the \textit{XMLbuilder} class. This class has one or two respond-messages per method that can be called on the server. Some has just got a single response-messages, because the only thing that changes is the parameters. Others have two respond-messages because they can out different results, e.g. a login-request can either be successful or fail. 

The XML builder is placed in the \textit{Logic Layer} because every response-message is specific for a piece of the game logic placed in the Game Thread. It is regarded as a part of the framework, but it has been used to develop the corresponding game and that can be seen in this class. The XML builder is not a necessary part to create, all response-message can be coded into the corresponding methods but it will affect the readability of the code and cluster a lot of string building into possibly complex methods. We chose to isolate the string building and created a separate class.

The \textit{XMLbuilder} is structured with a stringbuilder in each method. All methods for XML-building are formatted the same way in the code. If a XML-tag contains more information than an attribute, it will be split into a start-tag on one line, and an end-tag on another line. In between can be one or more tags, but if it only contains an attribute it will be a single line containing the start-tag, the attribute-value, and the end-tag. This makes the code easily readable when searching for the XML-formatting for a respond-message. 

The XML builder is compact and encapsulates all response-message very well. The class needs instantiation to be used, whereas it could arguably be more correct to make it as a static class. It is easy to create a new response-message, because already existing messages can be seen in the class.

We have two different standard formats for the XML strings. The first is a method call with attributes - the method name is used as an encapsulation with attributes as tags within. An example could be the response from the client entering a lobby. The response is encapsulated in the tags <Lobbyinfo></lobbyinfo>, with tags like <HostId> and <GameAlias>. The other standard format is a message from the server. An example could be the response to a login from the client. The response would be encapsulated in <Login></Login>, and contains a single tag called <Valid> with the value either true or false depending on the databases response.


%Data Layer
\subsection{Data Layer}
As described in \Cref{sec:server} the game treads and the admin class relies on a Data Mapper to handle communication with the database. The Data Mapper is based on the data mapper patten \fxwarning{ref WE bog}. The benefit of this is that the logic layer does not have to worry about the underlying database structure.
\subsubsection{Data Mapper} \label{sec:dbImplementation}
Handles reads, inserts, updates and deletes from the database. This is achieved by the exchange of domain objects between the logic layer and the data layer.
An example of this could be the Admin class requesting a list of all active games, the method for this takes no parameters, performs a lookup in game table, convert the result to domains objects and returns them as a list.
Another example could be a request for the a games a particular Account is taking part in. With the account id as parameter, the mapper performs lookups in the game and player (plays relation in the ER diagram \Cref{fig:ER}) tables, then converts it to a list of games.


\subsubsection{Domain Object} 
In order to represent the database in the logic layer we use Domain Objects. The domain objects are split into classes corresponding to tables in the database, though they do contain optional information based on foreign keys to other tables. An example could be a player domain object can contain information from the status effect table as well as the player table.

