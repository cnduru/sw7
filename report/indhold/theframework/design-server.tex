\subsection{Server Architecture}
\label{sec:server}

The server is structured in three layers: a dispatcher service, a logic layer and a data layer. This is based on the MVC Model described in \Cref{sec:mvc}. Because we wanted to develop a framework for creating location-based games, we had to make a decision how to achieve this goal. We could not interleave framework-specific functionality with the specific game we developed alongside our framework. Our solution was to create these three layers. We wanted to build a communication layer to handle the communication between the client and the server and a data layer to handle the communication between the server and the database. These two layers were chosen early in the design process. Our next choice was how to implement a specific game capable of using the framework, while still being scalable and encapsulated. We decided to make the threaded class called Game Thread handle all game-specific calculations, logics and responses. 
It was helpful to build a game alongside the framework. For example, this helped us to figure out the need for a service to handle timed events in the framework, which resulted in the Event Timer. We also discovered the need of a way to handle client requests that requested functionality prior to the Game Thread. This was requests like logging in, and seeing all active games, which cannot be handled by a game thread yet since it might not have been created yet. The result was the Admin. The Event Timer and the Admin are both positioned in the Logic Layer, because they help make logic decisions based on requests made by the client. 

%Server Architecture
\paragraph{Dispatcher Service}
The dispatcher is a service responsible for handling the communication with the client. It consists of an asynchronous I/O\fixme{IO or I/O throughout report?} and the dispatcher itself. Communication is done by an XML interface where requests from the client is verified by the dispatcher and passed to either the Admin module or a Game Thread depending on the message. 

\paragraph{Logic Layer}
This layer represents the business logic of the server application. The Admin module handles user creation, user login and game creation and is general for all game implementations. Game Thread is where specific game logic is implemented and its implementation will vary depending on the type of game. Event Timer is a general module handling events specified to happen at certain times. The game threads are responsible for creating events for themselves. The event timer will then notify a game thread at the specified time. As illustrated in \cref{fig:serverarch}, game threads are not a part of the framework, but can be ``plugged in''\fixme{change word??} to the framework. This increases the versatility of the game threads, since a developer can create his own game threads and plug them into the framework.

\paragraph{Data Layer}
Is the database representation in the model using the data mapper pattern\fxwarning{WE book ref}. The gateway is responsible for querying and updating the database and mapping the tables to objects used in the game logic and administration.

\begin{figure}[H]
  \centering
  \includegraphics[width=\textwidth]{billeder/serverarch.png}  
  \caption{Architecture of server displaying the branching of an incoming connection}
  \label{fig:serverarch}
\end{figure}

%Dispatcher Service
\subsection{Dispatcher Service}
This layer is responsible for communication between client and server, this responsibility is divided between an I/O module and a Dispatcher. In this section we will first examine Synchronous and Asynchronous I/O in order to determine which is best suited for this framework and then the Dispatcher which forwards messages to the Logic Layer. 

The Asynchronous I/O unit handles socket creation and TCP/IP communication, a rather trivial part of a client-server interface. The need for the Dispatcher unit was discovered by building our game. We need a way to dispatch requests from the client to the right component and method on the server. We dispatch to either Admin or Game Thread, by reading the request from the client and determine what type of request it is and continue accordingly. We decided to build the Dispatcher to encapsulate reading and analysing the requests from the client. Alternatively, we could have passed everything to Admin and incorporated the Dispatcher's functionality into the Admin, but we did not consider this a well structured framework.

\subsection{Blocking vs. non-blocking I/O}
For the client/server socket communication, a choice between blocking (synchronous) and non-blocking (asynchronous) I/O has to be taken.
% % blocking % %
Blocking I/O can have a better performance than non-blocking, but can cause problems when using a threaded architecture that spawns a new thread for each client.
% % non-blocking % %
Non-blocking I/O is chosen for this project. It scales well when there are many clients, and the system should scale well with an increasing number of clients. The ability to scale well does not come for free, however. Non-blocking IO is not always as fast as blocking IO and this can result in decreased performance. 
\subsubsection{Dispatcher}\label{subsec:dispatcherdesign}
The Dispatcher is a central component in the framework. It is placed in the \textit{Dispatcher Service} in the architecture, and is the first to interpret a request sent by a client. The dispatcher simply dispatches a call to the appropriate method with the appropriate parameters when it receives XML data. In order to do this, it reads the root tag of the XML to extract information to decide the appropriate destination and the right method to call. It attaches the XML as parameter to the method call.

The Dispatcher looks for the specified method call in the XML to pass to the Admin module. The Dispatcher then waits for the Admin to return an XML, which it returns to the asynchronous I/O.

The Dispatcher handles all requests to a game thread dynamically, as it only extracts the name of the correct method call before handing it over to the game thread. It then waits until the game thread returns an XML response string, which it returns to the asynchronous I/O.

%Logic Layer
\subsection{Logic Layer}
This layer parses and reacts to the messages received from the Dispatcher. The logic layer interact with the Game Threads and consists of the Event Timer module which is callable from the Game Threads and can do a callback at a later time. The layer also contains the Admin module responsible for creating Game threads as well as non-game specifics such as authenticating logins.

The framework allows the user to use services like logging a client in through an XML-based interface. Services in the framework requires the user to implement proper XML-based communication on the client-side. A discussion about the use of XML seen in \ref{subsec:interfaces}. The framework also offers to handle Game Events, timed to occur after a specified time-stamp. This service used by creating a Game Event in the Game Thread being build by the user of the framework. We decided to build the Event Timer this way to encapsulate time-handling, and make a generic module that can be used for many different events. The user of the framework can specify what should happen for a given event in the Game Event class.

\subsubsection{Admin}\label{subsec:admindesign}
The Admin module is responsible for handling server-requests not related to specifict Game Threads. The Admin module handles administrative calls like creating new games or accounts as well as verifying login requests. 

%impl?
%A \textit{CreateGame} call cannot be send to a game thread, obviously because the game thread has not yet been created. Therefore this method will create a game from a model-class \textit{Game}, store it in the database with the provided settings, start a thread on the server for it to run on. 


\subsubsection{Game Events}\label{subsec:gameEvents}
Game events are a data type which contains details about an event which should happen at some point in time in the game. It contains information about

\begin{itemize}
\item The type of event. This is user defined
\item ID of the game to which the event belongs
\item The time at which the event should be triggered
\end{itemize}

Game events are triggered by the Event Timer described in \cref{subsec:eventtimerdesign}

\subsubsection{Event Timer}\label{subsec:eventtimerdesign}
The event timer will run on a single thread, and handle executing game events at a time specified in the game event. All unexecuted game events are stored in a list. The event could be to place a shrine somewhere on the map. After an event has been triggered, it is removed from the list of game events. Game events are described in \cref{subsec:gameEvents}.

%Game Block
\subsection{Game Block}
This is the workstation of the user of the framework. This block interacts with all the layers in the framework and makes use of the functionality provided by the framework. This class encapsulates all game-specific logic.

This is where we designed the framework to allow the user to develop a game. The framework has several rules the user has to follow when using the framework. They are explained for each module in the design. An example is the XML communication from client to server.
 
\subsubsection{Game Thread}\label{designgamethread}
The game thread is designed to contain all game-specific logic for the game under development. This is the place where a user of the framework should design, develop and implement all the functionality for their specific game. The Game Thread is placed at the same level as the \textit{Logic Layer}, but is not part of the framework itself. 

The framework is designed to be running multiple instances of the Game Thread, one for each active game on the server. Each Game Thread will, as the name suggests, be running on its own unique thread. We decided to design the Game Thread this way to utilise the nature of modern computer systems, which runs multi-core systems allowing calculations on more threads simultaneously. By threading our program, we allow our server to make optimal use of its resources.

The framework has been designed to dynamically dispatch method calls to the Game Thread. It is important that the communication obeys the exact, and case sensitive game thread method-name. Otherwise the game thread will not be able to handle the provided request and the framework will fail. 

The Game Thread encapsulates all game-specific logic and is a manageable workstation for the user of the framework. It has the option to use the database through the \textit{Data Layer}, and the framework offers query-methods. Additionally it has the option to create Game Events that can be queued into the \textit{Event Timer}, which will perform changes in a game at specified timestamps. An example could be to spawn new objects.

The Game Thread will contain a constructor for creating new games, which will be called from the Admin class. This constructor will have to handle optional settings upon game-creation. The Game Thread will then call the database to update the state itself in the database according to the settings. Everything hereafter is optional and depends on the game the user wants to develop. When a game ends, a call to clean up the database will be initialized and the thread running the game will be stopped.

Since we develop a framework for location-based games, we end up with 2 categories of calls to the Game Thread. One is the location update, where the client provide a new position and get the position of nearby objects or players. The other category is client requests, where the client actively ask for some information or ask to perform an action within the game. The Game Thread will also contain the calculations for the game-logic handling these actions, whichever they may be depends on the game. 

%Data Layer
\subsection{Data Layer}
The data layer provides the user of the framework with the functionality to store data for later use. It was developed alongside our game and therefore provides the option to save location data, but it will also easily accommodate other game types.

\subsection{Database}\label{subsec:databasedesign}
In this project a database was needed to keep track of all information related to users and games they are playing. The database is designed with flexibility in mind which means that the logic behind a game is responsible for interpreting the data in the database. \figref{fig:ER}\fxwarning{export new ER diagram} shows the entity relation diagram for the database without attributes. The database is structured as follows:

\begin{figure}
  \centering
  \input{billeder/Server.tex}  
  \caption{ER Diagram.}
  \label{fig:ER}
\end{figure}

\paragraph{Account}
The Account entity represent the users in the system. An account can host games represented by the host relation and take part in games represented as owning a player in a particular game.

\paragraph{Player}
This entity represents a user playing in a game. A player can hold items, have status effects and be a member of a team represented by Inventory, has(Status Effect) and Team(is\_member) respectively. The player entity also contains score and location etc. for a particular player in a game.

\paragraph{Game}
This entity represent games either \textit{starting up}, \textit{in progress} or \textit{ended}.

\paragraph{Status Effect}
Effects on a player is represented by this entity. An effect could be \textit{disabled until time} on a particular player. Status Effects have an effect type which the game logic is responsible for defining.

\paragraph{Team}
Players can be members of teams in a game, though a player is not required to be on a team to allow free for all game modes.

\paragraph{Item}
Items represent any object in the inventory of players or on a location in the game world. Items can be anything from objects players can "pick up" to a capture-able area in the game world. The attributes and behavior of items are defined by game logic.

\paragraph{Location}
A location is an item in the game world that can belong to a team. To own a location, a team must take it first.





