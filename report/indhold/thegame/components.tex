\subsection{Components in the Application} \label{subsec:components}
As mentioned in \ref{sec:application_architecture}, the developed application is Android based. This means that most required components are readily available through the Android libraries. Activities has already been mentioned earlier, and is the base of the application. Other sizable components are however needed to provide the functionality that the application has to handle. These are listed below.

\paragraph{Activities}
The Android activity class is a fundamental component in programming Android applications\cite{android-activity}. In their basic form they consist of two components; a java class where one must override pre-defined methods, so they can be fitted for the programmers needs, along with an XML file describing the screen layout of that activity. To put it very roughly, the XML file defines visual layouts, while the activity class defines behavior on the layouts. As such, one can define components such as buttons, on-screen text and layouts such as lists in the XML file. These components can then be manipulated in the activity class, to manage functionality and behavior. A button defined in the XML will for example not work before coupled with an Android \textit{OnClickListener} which defines what happens when it is pressed. This has to be done in the activity class.

Android activities use a lifecycle model allowing the programmer to launch, pause (stop the activity, but keep it in the background) or destroy activities. This model allows for the activity flow described in \ref{sec:application_architecture}

\paragraph{Adapters}
Mentioned in \ref{subsec:patterns}, adapters are standard components in Android, meant to support lists, grids and the like. These are used in the \textit{ResumeGameActivity} and \textit{JoinGameActivity} to display a list of available games. In \textit{InvitePlayersActivity} an adapter is used to display a list of invited players. While the structure of the adapters is similar, the implementation is different each time. One has to fit the adapter to display each item as desired, provide correct functionality when using the lists and of course handle the data that is to be displayed in the list. The adapters are made public so that the activity that requires one can instantiate it, providing the required information (Layout and data to display)

\paragraph{LocationManager}
Given that the game should be location-based, something has to handle GPS locations. Again, Android provides a LocationManager class, customizable to fit the needs of the project. LocationManager is again a class made public. The \textit{GameActivity} can instantiate the LocationManager, which will get the current position if available, and otherwise the last known location. It is customized in that we choose how to handle whatever status the GPS is in, how often to attempt a GPS update and what to do with the GPS data. All the underlying functionality is however handled by Android.

\paragraph{Objects}
Juggling related data such as ID's of games and their names or players and their coordinates can be a hassle. The easy solution was to create classes to support these data, making them easier to keep track of. These objects are trivial in that they have some fields such as ID and name, and only provide methods to read/overwrite these variables.

Other than these components, there are those related to server communication. These should however not be seen as part of the application, but more as libraries created to access the server. These are is described in \ref{subsec:server-coupling}