\section{Game scenario}\label{sec:game}

This section describes the game we have decided to focus on. This game will be implemented as a prototype for proof of concept. It implements different uses of the server functionality that we want to display. It was also created for the purpose of setting a limitation for the server functionality. Otherwise we could have kept creating functionality for multiple different purposes. The game we have decided to create a scenario for, is a game based on the real-life coordinates of the participants. The game should be playable among friends, and be interactive. The game is a prototype that could be implemented on top of our server and item balance, e.g. a certain weapon may be stronger than intended, is in no way to be considered final in the game.

The biggest advantage of creating a specific game scenario is a better insight into the general functionality needed on the server side. It furthermore provides a goal which helps with motivation and work culture, in the sense that we could more easily divide the work needed in the project. One of the drawbacks of using a scenario in this way is that, if one is not not careful, it could cause us to accidentally focus strictly on the scenario, and might exclude functionality that could be useful on the server side. Just because the functionality does not fit a certain scenario does not mean it is not useful for a general model. Furthermore, we might not get the flexibility we look for in a server as we don't have different types of games to test it with.

\subsection{Game Rules}\label{subsec:game-rules}

The following is a description of the game scenario.

\paragraph{Game winning criteria}
The idea is for one of the competing teams to obtain an ultimate relic. Once a team locates it, the game should end with a victory for the team with the relic. At the start of the game it is impossible to know where the relic is located, and you gradually locate it by finding clues around the map.
Another way to win is to be the team with the largest amount of points if a time limit is set and reached.

\paragraph{Objectives}
There is a list of different objectives each with a unique functionality and meaning to the game.
\begin{itemize}
\item Point Objectives - Gives +1 team point, takes 30 seconds to capture and reveals your locations to nearby enemies for 120 seconds.
\item Puzzle Objectives - Gives your team a clue, takes 300 seconds to capture and reveals your location for 600 seconds.
\item Crates - Can contain weapons, instantly acquired and reveals your location for 120 seconds.
\item Powerups - Can contain bonus for your stats,  instantly acquired and reveals your location for 120 seconds.
\item Shrines - Will give different bonuses depending on the shrine type, the shrine is active for 4 hours after activation, takes 300 seconds to capture and reveals your location for 600 seconds.
\item Ultimate relic - Digging for the ultimate relic has a one hour cooldown, takes 300 seconds to capture and reveals your location for 600 seconds.
\item Watchtowers - Gives increased range when within the tower.
\end{itemize}

\paragraph{Items}
There is a list of different items in the game as well. These items are obtainable through crates dropping.
\begin{itemize}
\item Start-pistol - Range: 25 meter, Damage: 20 HP, reveal-time: 60 seconds and cooldown: 20 seconds.
\item Shotgun - Range: 15 meter, Damage: 60 HP, reveal-time: 90 seconds and cooldown: 30 seconds.
\item Sniper - Range: 40-80 meter, Damage: 70 HP, reveal-time: 90 seconds and cooldown: 120 seconds.
\item Ammunition - Pistol ammunition +16 bullets, Shotgun ammunition +4 shells and Sniper Ammunition +3 bullets.
\item Medkit - gives +25 health points on pick up.
\end{itemize}

\paragraph{Powerups}
This is the list of what is obtainable as powerups.
\begin{itemize}
\item Armor - +15 Armor (Max 50 Armor)
\item Vision - +2 meter (Max 50 Meter)
\item Scan - +20 meter (Max 60 Meter)
\item Range - +2\% range (Max 6\% Range)
\item Shield - Blocks the next shot
\end{itemize}

\paragraph{Shrines}
The following is a list of the different shrines and functionality. Shrines expire after a given amount of hours.
\begin{itemize}
\item Power Plant - Provides 2 points every half an hour.
\item Locater - Pinpoints the location of an object in range of scan every half an hour.
\item Shield Generator - Gives the entire team a shield every hour.
\item Factory - Provides each user on the team with 3 pistol bullets, 2 shotgun shells and 1 sniper bullet every half an hour.
\end{itemize}

\subsection{Different game scenarios}\label{subsec:game-scenarios}
Since we created a server with functionality completely independent of the game implemented on top \fxnote{reference til separate frameworks fra server shit}, a lot of different games can be implemented. The way we have chosen is a pretty straight forward method with lots of different shrines and items.

Something simple that could be implemented fast could be a multiplyer pac-man game. Each game would feature one pac-man and several different ghosts. The objective of the game is for the pac-man to collect all the pellets (which could easily be implemented the same way point objectives), and the fruit that makes pac-man chase the ghost would be crates. The server would get fed coordinates and possibly have a live feed of the map on the client side.
Along those lines a multiplayer game like assassins creed could be emulated. A game happening in the Renaissance (with related weapons like crossbows and swords), where each player is a target of another player - and the winner would be guy to first kill his target.

The item system in the database is extremely flexible - they could be weapons as in our example but fit for a certain era or time period. It is completely up to the developer to determine how the weapons are implemented. One could imagine having a theme like the wild west, where everyone only have six-shooters or a close combat game where people only have melee weapons. 

Furthermore the relation between games and teams are 1-N, and this results in us being able to have co-op or free for all games. A king of the hill game seems like a suitable implementation of this, trying to hold shrines while everyone is out to kill you. This means that it would be possible to implement several other popular game-modes like last man standing or deathmatch where you respawn after being killed.

It is also possible to refrain completely from fighting games. An example could be a gather game similar to Googles Ingress, where you get points for having control of certain areas of the map.

There is a lot of different games that easily could be implemented on top of our framework.



